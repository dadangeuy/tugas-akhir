\chapter{PENGUJIAN DAN EVALUASI}
\par Bab ini membahas pengujian dan evaluasi terhadap perangkat lunak yang telah diimplementasikan, dengan menggunakan metode \textit{blackbox}.

\section{Lingkungan Pengujian}
\par Lingkungan yang digunakan untuk menguji tugas akhir ini memiliki spesifikasi perangkat keras dan lunak yang ditunjukkan pada tabel \ref{5:tabel_spesifikasi_server}, \ref{5:tabel_spesifikasi_perangkat_android}, dan \ref{5:tabel_spesifikasi_perangkat_ios}.
\begin{longtable}{|p{2.5cm}|p{6.5cm}|}
    \hline
    \textbf{Jenis} & \textbf{Spesifikasi} \\ \hline
    CPU & - \\ \hline
    Memory & - \\ \hline
    Sistem Operasi & - \\ \hline
    \caption{Spesifikasi Server}
    \label{5:tabel_spesifikasi_server}
\end{longtable}
\begin{longtable}{|p{2.5cm}|p{6.5cm}|}
    \hline
    \textbf{Jenis} & \textbf{Spesifikasi} \\ \hline
    CPU & Snapdragon 636 \\ \hline
    Memory & 3 GB \\ \hline
    Sistem Operasi & Android 9 (Pie) \\ \hline
    \caption{Spesifikasi Perangkat Android}
    \label{5:tabel_spesifikasi_perangkat_android}
\end{longtable}
\begin{longtable}{|p{2.5cm}|p{6.5cm}|}
    \hline
    \textbf{Jenis} & \textbf{Spesifikasi} \\ \hline
    CPU & Apple A10 Fusion \\ \hline
    Memory & 3 GB \\ \hline
    Sistem Operasi & iOS 12 \\ \hline
    \caption{Spesifikasi Perangkat iOS}
    \label{5:tabel_spesifikasi_perangkat_ios}
\end{longtable}

\section{Pengujian Fungsionalitas}

\subsection{Pengujian Pembuatan Packet}
\begin{longtable}{|p{2.5cm}|p{6.5cm}|}
    \hline
    \textbf{No. Pengujian} & UJ01 \\ \hline
    \textbf{Kode Kasus Penggunaan} & UC01 \\ \hline
    \textbf{Nama} & Pengujian Pembuatan Packet \\ \hline
    \textbf{Aktor} & Scheduler \\ \hline
    \textbf{Tujuan Pengujian} & Menguji fungsionalitas untuk pembuatan packet. \\ \hline
    \textbf{Kondisi Awal} & - \\ \hline
    \textbf{Data Uji} & - \\ \hline
    \textbf{Langkah Pengujian} & - \\ \hline
    \textbf{Hasil yang diharapkan} & - \\ \hline
    \textbf{Hasil yang didapatkan} & - \\ \hline
    \textbf{Hasil Pengujian} & - \\ \hline
    \textbf{Kondisi Akhir} & - \\ \hline
    \caption{Pengujian Pembuatan Packet}
    \label{5:tabel_pengujian_pembuatan_packet}
\end{longtable}

\subsection{Pengujian Menambahkan Packet ke Antrian}

\subsection{Pengujian Pengiriman Packet ke APNs}

\subsection{Pengujian Pengiriman Packet ke FCM}

\section{Pengujian Performa}

\subsection{Pengujian Pengiriman Notifikasi ke 1.000 Perangkat}

\subsection{Pengujian Pengiriman Notifikasi ke 10.000 Perangkat}

\subsection{Pengujian Pengiriman Notifikasi ke 100.000 Perangkat}

\subsection{Pengujian Pengiriman Notifikasi ke 1.000.000 Perangkat}

\section{Evaluasi Hasil Pengujian}

\subsection{Evaluasi Pengujian Fungsionalitas}

\subsection{Evaluasi Pengujian Performa}
