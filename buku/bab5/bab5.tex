\chapter{PENGUJIAN DAN EVALUASI}
\par Bab ini membahas pengujian dan evaluasi terhadap perangkat lunak yang telah diimplementasikan, dengan menggunakan metode \textit{blackbox}.

\section{Lingkungan Pengujian}
\par Lingkungan yang digunakan untuk menguji tugas akhir ini memiliki spesifikasi perangkat keras dan lunak yang ditunjukkan pada tabel \ref{5:tabel_spesifikasi_server}, \ref{5:tabel_spesifikasi_perangkat_android}, dan \ref{5:tabel_spesifikasi_perangkat_ios}.
\begin{longtable}{|p{2.5cm}|p{6.5cm}|}
	\caption{Spesifikasi Server} \label{5:tabel_spesifikasi_server} \\ \hline
    \textbf{Jenis} & \textbf{Spesifikasi} \\ \hline
    CPU & Intel(R) Xeon(R) CPU E5-2690 v4 \\ \hline
    CPU Core & 4 \\ \hline
    Memory & 8 GB \\ \hline
    Sistem Operasi & Ubuntu 18.04 \\ \hline
\end{longtable}
\begin{longtable}{|p{2.5cm}|p{6.5cm}|}
	\caption{Spesifikasi Perangkat Android} \label{5:tabel_spesifikasi_perangkat_android} \\ \hline
    \textbf{Jenis} & \textbf{Spesifikasi} \\ \hline
    CPU & Snapdragon 636 \\ \hline
    Memory & 3 GB \\ \hline
    Sistem Operasi & Android 9 (Pie) \\ \hline
\end{longtable}
\begin{longtable}{|p{2.5cm}|p{6.5cm}|}
	\caption{Spesifikasi Perangkat iOS} \label{5:tabel_spesifikasi_perangkat_ios} \\ \hline
    \textbf{Jenis} & \textbf{Spesifikasi} \\ \hline
    CPU & Apple A10 Fusion \\ \hline
    Memory & 3 GB \\ \hline
    Sistem Operasi & iOS 12 \\ \hline
\end{longtable}

\section{Pengujian Fungsional}

\subsection{Pengujian Pembuatan Packet}

\begin{longtable}{|p{2.5cm}|p{6.5cm}|}
	\caption{Pengujian Pembuatan Packet} \label{t:uji_pembuatan_packet} \\ \hline
	\textbf{Kode} & FT-01 \\ \hline
	\textbf{Nama} & Pengujian Pembuatan Packet \\ \hline
	\textbf{Tujuan} & Menguji apakah sistem mampu membuatkan data \textit{packet} dari data \textit{batch} \\ \hline
	\textbf{Kondisi Awal} & Data batch belum ditambahkan ke sistem basis data \\ \hline
	\textbf{Langkah Pengujian} &  
	\begin{enumerate}
		\item Pengguna menambahkan data batch baru lewat halaman manajemen aplikasi push notification terpusat.
		\item Sistem akan menyimpan data \textit{batch} ke dalam sistem basis data
		\item Scheduler akan mengecek \textit{batch} yang baru dibuat pada sistem basis data
		\item Pada setiap batch, Scheduler akan mencari device yang sesuai dengan pengguna pada group \textit{batch} tersebut
		\item Untuk setiap \textit{device}, Scheduler membuatkan \textit{packet} dan menyimpannya di database.
	\end{enumerate} \\ \hline
	\textbf{Hasil yang diharapkan} & Data \textit{packet} tersimpan di sistem basis data \\ \hline
	\textbf{Hasil yang diperoleh} & Data \textit{packet} tersimpan di sistem basis data \\ \hline
	\textbf{Hasil pengujian} & Berhasil \\ \hline
\end{longtable}

\subsection{Pengujian Menambahkan Packet ke Antrian}
\begin{longtable}{|p{2.5cm}|p{6.5cm}|}
	\caption{Pengujian Pembuatan Antrian} \label{t:uji_pembuatan_antrian} \\ \hline
	\textbf{Kode} & FT-02 \\ \hline
	\textbf{Nama} & Pengujian Pembuatan Antrian \\ \hline
	\textbf{Tujuan} & Menguji apakah sistem mampu membuatkan data antrian dari data \textit{packet} \\ \hline
	\textbf{Kondisi Awal} & Data packet sudah dibuat dan belum ditambahkan ke antrian \\ \hline
	\textbf{Langkah Pengujian} &  
	\begin{enumerate}
		\item Scheduler mencari paket yang belum dan sudah waktunya dikirim
		\item Scheduler memperbarui data \textit{packet} dari "Dibuat" menjadi "Menunggu"
		\item Scheduler akan mengecek apakah \textit{device} dari \textit{packet} itu adalah untuk perangkat Android, iOS atau web.
		\begin{enumerate}
			\item Apabila \textit{device} untuk perangkat Android, Scheduler akan mengirim ke topik antrian "android"
			\item Apabila \textit{device} untuk perangkat iOS, Scheduler akan mengirim ke topik antrian "ios"
			\item Apabila \textit{device} untuk perangkat web, Scheduler akan mengirim ke topik antrian "web"
		\end{enumerate}
	\end{enumerate} \\ \hline
	\textbf{Hasil yang diharapkan} & Data antrian tersimpan di Kafka. \\ \hline
	\textbf{Hasil yang diperoleh} & Data antrian tersimpan di Kafka. \\ \hline
	\textbf{Hasil pengujian} & Berhasil \\ \hline
\end{longtable}

\subsection{Pengujian Pengiriman Packet ke APNs}
\begin{longtable}{|p{2.5cm}|p{6.5cm}|}
	\caption{Pengujian Pengiriman Packet ke APNs} \label{t:uji_pengiriman_packet_apn} \\ \hline
	\textbf{Kode} & FT-03 \\ \hline
	\textbf{Nama} & Pengujian Pembuatan Packet \\ \hline
	\textbf{Tujuan} & Menguji apakah sistem mampu mengirim \textit{packet} ke APNs \\ \hline
	\textbf{Kondisi Awal} &  Packet ada pada antrian topik "ios"\\ \hline
	\textbf{Langkah Pengujian} &  
	\begin{enumerate}
		\item Kafka mengirim \textit{packet} pada antrian topik "ios" ke Sender APN
		\item Sender APN menerima packet dari antrian topik "ios"
		\item Sender APN mengirim \textit{request push notification} ke APNs
		\begin{enumerate}
			\item Apabila berhasil, Sender APN memperbarui data \textit{packet} menjadi "sukses"
			\item Apabila gagal, Sender APN memperbarui data \textit{packet} menjadi "gagal"
		\end{enumerate}
	\end{enumerate} \\ \hline
	\textbf{Hasil yang diharapkan} & Notifikasi terkirim pada perangkat iOS pengguna \\ \hline
	\textbf{Hasil yang diperoleh} & Notifikasi terkirim pada perangkat iOS pengguna \\ \hline
	\textbf{Hasil pengujian} & Berhasil \\ \hline
\end{longtable}

\subsection{Pengujian Pengiriman Packet ke FCM}
\begin{longtable}{|p{2.5cm}|p{6.5cm}|}
	\caption{Pengujian Pengiriman Packet ke FCM} \label{t:uji_pengiriman_packet_fcm} \\ \hline
	\textbf{Kode} & FT-04 \\ \hline
	\textbf{Nama} & Pengujian Pengiriman Packet ke FCM \\ \hline
	\textbf{Tujuan} & Menguji apakah sistem mampu mengirim \textit{packet} ke FCM \\ \hline
	\textbf{Kondisi Awal} &  Packet ada pada antrian topik "android" atau "web"\\ \hline
	\textbf{Langkah Pengujian} &  
	\begin{enumerate}
		\item Kafka mengirim \textit{packet} pada antrian topik "android" dan "web" ke Sender FCM
		\item Sender FCM menerima packet dari antrian topik "android" atau "web"
		\item Sender FCM mengirim \textit{request push notification} ke FCM
		\begin{enumerate}
			\item Apabila berhasil, Sender FCM memperbarui data \textit{packet} menjadi "sukses"
			\item Apabila gagal, Sender FCM memperbarui data \textit{packet} menjadi "gagal"
		\end{enumerate}
	\end{enumerate} \\ \hline
	\textbf{Hasil yang diharapkan} & Notifikasi terkirim pada perangkat Android atau web pengguna \\ \hline
	\textbf{Hasil yang diperoleh} & Notifikasi terkirim pada perangkat Android atau web pengguna \\ \hline
	\textbf{Hasil pengujian} & Berhasil \\ \hline
\end{longtable}

\subsection{Pengujian Menampilkan Penggunaan Sumber Daya}
\begin{longtable}{|p{2.5cm}|p{6.5cm}|}
	\caption{Pengujian Menampilkan Penggunaan Sumber Daya} \label{t:uji_menampilkan_penggunaan_sumber_daya} \\ \hline
	\textbf{Kode} & FT-05 \\ \hline
	\textbf{Nama} & Pengujian Menampilkan Penggunaan Sumber Daya \\ \hline
	\textbf{Tujuan} & Menguji apakah sistem mampu menampilkan penggunaan sumber daya \\ \hline
	\textbf{Kondisi Awal} &  \textit{Server} dalam keadaan berjalan\\ \hline
	\textbf{Langkah Pengujian} &  
	\begin{enumerate}
		\item Aplikasi Manajemen mengakses \textit{endpoint} untuk menampilkan penggunaan sumber daya
		\item Sistem mengirim response berupa JSON mengenai penggunaan sumber daya
	\end{enumerate} \\ \hline
	\textbf{Hasil yang diharapkan} & Aplikasi dapat mengetahui penggunaan sumber daya \\ \hline
	\textbf{Hasil yang diperoleh} & Aplikasi dapat mengetahui penggunaan sumber daya \\ \hline
	\textbf{Hasil pengujian} & Berhasil \\ \hline
\end{longtable}
\subsection{Pengujian Menampilkan Status Kesehatan}
\begin{longtable}{|p{2.5cm}|p{6.5cm}|}
	\caption{Pengujian Menampilkan Status Kesehatan} \label{t:uji_menampilkan_status_kesehatan} \\ \hline
	\textbf{Kode} & FT-06 \\ \hline
	\textbf{Nama} & Pengujian Menampilkan Status Kesehatan \\ \hline
	\textbf{Tujuan} & Menguji apakah sistem mampu menampilkan status kesehatan \\ \hline
	\textbf{Kondisi Awal} &  \textit{Server} dalam keadaan berjalan\\ \hline
	\textbf{Langkah Pengujian} &  
	\begin{enumerate}
		\item Aplikasi Manajemen mengakses \textit{endpoint} untuk menampilkan status kesehatan
		\item Sistem mengirim response berupa JSON mengenai status kesehatan
	\end{enumerate} \\ \hline
	\textbf{Hasil yang diharapkan} & Aplikasi dapat mengetahui status kesehatan \\ \hline
	\textbf{Hasil yang diperoleh} & Aplikasi dapat mengetahui status kesehatan \\ \hline
	\textbf{Hasil pengujian} & Berhasil \\ \hline
\end{longtable}
\subsection{Pengujian Menampilkan Konfigurasi}
\begin{longtable}{|p{2.5cm}|p{6.5cm}|}
	\caption{Pengujian Menampilkan Konfigurasi} \label{t:uji_menampilkan_konfigurasi} \\ \hline
	\textbf{Kode} & FT-07 \\ \hline
	\textbf{Nama} & Pengujian Menampilkan Konfigurasi \\ \hline
	\textbf{Tujuan} & Menguji apakah sistem mampu menampilkan konfigurasi \\ \hline
	\textbf{Kondisi Awal} &  \textit{Server} dalam keadaan berjalan\\ \hline
	\textbf{Langkah Pengujian} &  
	\begin{enumerate}
		\item Aplikasi Manajemen mengakses \textit{endpoint} untuk menampilkan konfigurasi
		\item Sistem mengirim response berupa JSON mengenai konfigurasi
	\end{enumerate} \\ \hline
	\textbf{Hasil yang diharapkan} & Aplikasi dapat mengetahui konfigurasi \\ \hline
	\textbf{Hasil yang diperoleh} & Aplikasi dapat mengetahui konfigurasi \\ \hline
	\textbf{Hasil pengujian} & Berhasil \\ \hline
\end{longtable}
\subsection{Pengujian Menampilkan Log}
\begin{longtable}{|p{2.5cm}|p{6.5cm}|}
	\caption{Pengujian Menampilkan Log} \label{t:uji_menampilkan_log} \\ \hline
	\textbf{Kode} & FT-08 \\ \hline
	\textbf{Nama} & Pengujian Menampilkan Log \\ \hline
	\textbf{Tujuan} & Menguji apakah sistem mampu menampilkan log \\ \hline
	\textbf{Kondisi Awal} &  \textit{Server} dalam keadaan berjalan\\ \hline
	\textbf{Langkah Pengujian} &  
	\begin{enumerate}
		\item Aplikasi Manajemen mengakses \textit{endpoint} untuk menampilkan log
		\item Sistem mengirim response berupa JSON mengenai log
	\end{enumerate} \\ \hline
	\textbf{Hasil yang diharapkan} & Aplikasi dapat mengetahui log \\ \hline
	\textbf{Hasil yang diperoleh} & Aplikasi dapat mengetahui log \\ \hline
	\textbf{Hasil pengujian} & Berhasil \\ \hline
\end{longtable}
\section{Pengujian Non Fungsional}

\subsection{Pengujian Performa}
\par Lorem Ipsum
\begin{longtable}{|p{1.3cm}|p{1.3cm}|p{1.3cm}|p{1.8cm}|p{1.8cm}|p{1.8cm}|}
	\caption{Hasil Pengujian Performa} \label{t:performa} \\ \hline
	\rowcolor{gray!10} & \multicolumn{2}{c|}{Jumlah Perangkat} & \multicolumn{3}{c|}{Waktu Pengolahan \textit{Packet}} \\ \hhline{~|*5{-}|}
	\rowcolor{gray!10} \multirow{-2}{*}{Kode} & Android & iOS & Pembuatan & Pengiriman & Total \\ \hline
	NFT-01 & 1.000 & 1.000 & 00:00:04 & 00:02:13 & 00:02:17 \\ \hline
	NFT-02 & 10.000 & 10.000 & 00:00:25 & 00:12:09 & 00:12:34 \\ \hline
	NFT-03 & 100.000 & 100.000 & 00:03:46 & 01:57:46 & 02:01:32 \\ \hline
\end{longtable}

\subsection{Pengujian Keandalan}
\par Lorem Ipsum
\begin{longtable}{|p{1.3cm}|p{3cm}|p{1.3cm}|p{1.5cm}|}
	\caption{Hasil Pengujian Keandalan} \label{t:performa} \\ \hline
	\rowcolor{gray!10} &  & \multicolumn{2}{c|}{Packet Terkirim} \\ \hhline{~|~|*2{-}|}
	\rowcolor{gray!10} \multirow{-2}{*}{Kode} & \multirow{-2}{*}{Jumlah Perangkat} & APNs & iOS \\ \hline
	NFT-04 & 1.000 & 1.000 & - \\ \hline
	NFT-05 & 10.000 & 10.000 & - \\ \hline
	NFT-06 & 100.000 & 100.000 & - \\ \hline
\end{longtable}

\subsection{Pengujian Ketersediaan}

\subsection{Pengujian Durabilitas}

\section{Evaluasi Hasil Pengujian}

\subsection{Evaluasi Pengujian Fungsional}

\subsection{Evaluasi Pengujian Non Fungsional}
