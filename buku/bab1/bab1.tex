\vspace{0ex}
\chapter {PENDAHULUAN}
\par Bab ini menjelaskan konteks tugas akhir yang akan dikerjakan, termasuk latar belakang, rumusan masalah, batasan masalah, tujuan, manfaat, metodologi, dan sistematika penulisan.

\section{Latar Belakang}
\par \textit{Push notification} adalah pesan yang muncul di perangkat pengguna lewat aplikasi yang terpasang dalam perangkat tersebut. \textit{Push notification} pada umumnya digunakan untuk menyebarkan informasi seperti promo, pengingat pembayaran, berita terbaru, dan sebagainya.
\par Pengiriman \textit{push notification} di Institut Teknologi Sepuluh Nopember menggunakan layanan aplikasi push notification terpusat. Aplikasi push notification terpusat adalah layanan pengiriman \textit{push notification} berbasis web yang dikembangkan untuk memudahkan manajemen dan proses penyebaran informasi ke pengguna aplikasi yang ada di Institut Teknologi Sepuluh Nopember \cite{application-thesis}.
\par Aplikasi push notification terpusat menggunakan metode \textit{message queue} dalam proses pengiriman \textit{push notification}. \textit{Message queue} adalah metode komunikasi antar sistem secara asynchronous dengan cara membuat antrian untuk pesan-pesan yang akan diolah \cite{message-queue-online}. Message queue pada umumnya digunakan oleh aplikasi yang membutuhkan waktu lama untuk mengolah data, seperti pengiriman email, transaksi perbankan, dan sebagainya.
\par \textit{Message queue} yang digunakan oleh aplikasi push notification terpusat memiliki kelemahan dari sisi keandalan pengiriman \textit{push notification}. Berdasarkan hasil pengujian, untuk kasus uji pengiriman \textit{push notification} ke 3000 perangkat, tingkat keberhasilan pengiriman hanya 63,8 persen \cite{application-thesis}.
\par Oleh karena itu, pada tugas akhir ini penulis mengusulkan untuk menggunakan Kafka sebagai pengganti message queue yang saat ini digunakan. Kafka adalah layanan terdistribusi untuk \textit{data streaming} yang dibuat berdasarkan konsep \textit{message queue} \cite{kafka-online}. Berdasarkan hasil uji yang dilakukan oleh pengembang lain, Kafka terbukti mampu mengirimkan push notification ke 1 juta perangkat dengan tingkat keberhasilan pengiriman 100 persen \cite{prototype-article}. Selain perbaikan \textit{message queue}, penulis juga akan menambahkan fitur \textit{monitoring} untuk memudahkan pengelola aplikasi mendiagnosa jika terdapat masalah pada aplikasi push notification terpusat.

\section {Rumusan Masalah}
Rumusan masalah yang diangkat pada tugas akhir ini adalah sebagai berikut:
\begin {enumerate}
\item Bagaimana konfigurasi Kafka agar dapat digunakan oleh aplikasi push notification terpusat?
\item Bagaimana implementasi \textit{message queue} dengan Kafka pada aplikasi push notification terpusat?
\item Bagaimana implementasi penjadwalan pengiriman \textit{push notification} pada aplikasi push notification terpusat?
\item Bagaimana implementasi pengiriman \textit{push notification} ke perangkat pengguna?
\item Bagaimana implementasi \textit{monitoring} pada aplikasi push notification terpusat?
\end {enumerate}

\section {Batasan Masalah}
Permasalahan yang dibahas pada tugas akhir ini memiliki batasan sebagai berikut:
\begin {enumerate}
\item Aplikasi push notification terpusat digunakan untuk ruang lingkup Institut Teknologi Sepuluh Nopember.
\item Aplikasi push notification terpusat mengirim \textit{push notification} ke perangkat android, web, dan ios dengan menggunakan layanan Apple Push Notification Service dan Firebase Cloud Messaging.
\item \textit{Monitoring} digunakan untuk melihat kondisi layanan aplikasi push notification terpusat.
\item Teknologi yang digunakan dalam pembuatan aplikasi ini adalah bahasa pemrograman Java dengan kerangka kerja Spring.
\end {enumerate}

\section {Tujuan}
Tujuan yang akan dicapai dalam tugas akhir ini adalah sebagai berikut:
\begin{enumerate}
	\item Meningkatkan keberhasilan pengiriman \textit{push notification}.
	\item Memudahkan proses pemantauan aplikasi.
\end{enumerate}

\section{Manfaat}
Tugas akhir ini diharapkan dapat mengembangkan aplikasi push notification terpusat, sehingga dapat menjadi layanan penyebaran informasi yang dapat diandalkan dan dapat memudahkan pengelola untuk memantau kondisi layanan.

\section {Metodologi}
Pembuatan tugas akhir ini dilakukan dengan menggunakan metodologi sebagai berikut:
\begin{enumerate}
\item Penyusunan proposal\\
Pada tahapan ini, penulis akan menyusun rencana dan langkah-langkah yang akan dilakukan dalam proses pembuatan tugas akhir.
\item Studi literatur\\
Pada tahap ini, akan dicari studi literatur yang relevan untuk dijadikan referensi dalam pengerjaan tugas akhir, antara lain mengenai Java, Spring, Firebase Cloud Messaging, Apple Push Notification Service, dan Apache Kafka.
\item Analisis dan Perancangan Sistem\\
Tahap ini meliputi perumusan kebutuhan fungsional, kebutuhan non-fungsional, kasus penggunaan, diagram aktivitas, diagram kelas, dan diagram sekuens.
\item Implementasi\\
Aplikasi ini diimplementasikan dengan menggunakan kakas bantu sebagai berikut:
\begin{enumerate}
\item Sistem basis data menggunakan Microsoft SQL Server.
\item Message queueing menggunakan Apache Kafka.
\item Aplikasi dibuat menggunakan bahasa pemrograman Java dengan kerangka kerja Spring.
\item Pembuatan aplikasi menggunakan IDE Intellij IDEA.
\end{enumerate}
\item Uji Coba dan Evaluasi\\
Pengujian dilakukan dengan mengirimkan push notification dalam jumlah besar untuk mengetahui tingkat keberhasilan dan kecepatan pengiriman notifikasi dengan model pengujian \textit{blackbox} testing.
\item Penyusunan buku\\
Pada tahap ini dilakukan penyusunan laporan yang menjelaskan dasar teori dan metode yang digunakan dalam tugas akhir ini serta hasil dari implementasi aplikasi perangkat lunak yang telah dibuat.
\end{enumerate}

\section {Sistematika Penulisan}
Sistematika penulisan yang digunakan dalam laporan tugas akhir ini adalah sebagai berikut:
\begin{enumerate}
\item Bab 1 : PENDAHULUAN
Bab ini menjelaskan konteks tugas akhir yang akan dikerjakan, termasuk latar belakang, rumusan masalah, batasan masalah, tujuan, manfaat, metodologi, dan sistematika penulisan.
\item Bab 2 : DASAR TEORI
Bab ini menjelaskan dasar teori yang akan digunakan dalam proses pengerjaan tugas akhir.
\item Bab 3 : ANALISIS DAN PERANCANGAN SISTEM
Bab ini menjelaskan tentang analisis permasalahan, deskripsi umum sistem, spesifikasi kebutuhan perangkat lunak, lingkungan perancangan, perancangan arsitektur sistem, dan diagram kelas berdasarkan dasar teori yang dijelaskan pada bab 2.
\item Bab 4 : IMPLEMENTASI SISTEM
Bab ini menjelaskan implementasi dari perancangan dan implementasi fitur-fitur penunjang aplikasi yang dijelaskan pada bab 3.
\item Bab 5 : PENGUJIAN DAN EVALUASI
Bab ini menjelaskan hasil pengujian dan evaluasi dengan metode \textit{blackbox} untuk mengetahui nilai fungsionalitas dari perangkat lunak yang telah diimplementasikan pada bab 4.
\item Bab 6 : PENUTUP
Bab ini berisi kesimpulan yang telah didapat dari hasil pengujian dan evaluasi yang telah dilakukan.
\end{enumerate}
