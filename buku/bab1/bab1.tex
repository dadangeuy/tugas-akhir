\vspace{0ex}
\chapter {PENDAHULUAN}
\par Bab ini menjelaskan konteks tugas akhir yang akan dikerjakan, termasuk latar belakang, rumusan masalah, batasan masalah, tujuan, manfaat, metodologi, dan sistematika penulisan.

\section{Latar Belakang}
\par Aplikasi \textit{push notification} terpusat merupakan layanan pengiriman notifikasi berbasis \textit{web} \cite{application-thesis}. Aplikasi ini dikembangkan dengan tujuan untuk mempermudah manajemen pengiriman notifikasi ke perangkat \textit{Android}, \textit{Web} dan \textit{iOS} di lingkup Institut Teknologi Sepuluh Nopember \cite{application-thesis}.
\par Untuk menangani pengiriman notifikasi dalam jumlah besar, aplikasi \textit{push notification} terpusat menerapkan konsep \textit{message queueing} yang diimplementasikan langsung dalam aplikasi oleh pengembang sebelumnya. Solusi ini memiliki kelemahan dari sisi keandalan pengiriman notifikasinya. Berdasarkan hasil pengujian, untuk kasus uji pengiriman \textit{push notification} ke 3000 perangkat pengguna, tingkat keberhasilan pengiriman hanya 63,8 persen dengan durasi waktu pengiriman sekitar 3 jam \cite{application-thesis}.
\par Oleh karena itu, pada tugas akhir ini aplikasi \textit{push notification} terpusat akan dikembangkan dengan mengimplementasikan jurnal \textit{“A Prototype Framework for High Performance Push Notifications”} \cite{prototype-article}. Berdasarkan hasil pengujian, aplikasi \textit{push notification} yang dibuat dengan menggunakan \textit{Apache Kafka} sebagai \textit{message queueing}-nya mampu mengirimkan \textit{push notification} ke 1 juta perangkat pengguna dengan tingkat keberhasilan pengiriman 100 persen dengan durasi waktu pengiriman sekitar 1 jam \cite{prototype-article}.

\section {Rumusan Masalah}
Permasalahan yang akan diselesaikan pada tugas akhir ini adalah sebagai berikut:
\begin {enumerate}
\item Bagaimana meningkatkan tingkat keberhasilan pengiriman notifikasi pada aplikasi push notification terpusat?
\item Bagaimana mengimplementasikan message queueing push notification dengan menggunakan Apache Kafka untuk meningkatkan kecepatan pengiriman?
\item Bagaimana mengimplementasikan aplikasi untuk monitoring aplikasi push notification terpusat?
\end {enumerate}

\section {Batasan Masalah}
Batasan dari masalah yang akan diselesaikan adalah sebagai berikut:
\begin {enumerate}
\item Aplikasi push notification terpusat digunakan untuk ruang lingkup Institut Teknologi Sepuluh Nopember.
\item Aplikasi push notification terpusat mengirim pesan ke perangkat Android dan iOS lewat layanan Firebase Cloud Messaging dan Apple Push Notification Service.
\item Aplikasi monitor digunakan untuk memantau kondisi layanan aplikasi push notification terpusat.
\item Teknologi yang digunakan dalam pembuatan aplikasi ini adalah bahasa pemrograman Java dengan kerangka kerja Spring.
\end {enumerate}

\section {Tujuan}
Tujuan pembuatan tugas akhir ini adalah :
\begin{enumerate}
	\item untuk meningkatkan tingkat keberhasilan push notification dari aplikasi push notification terpusat
	\item untuk meningkatkan kecepatan pengiriman push notification dari aplikasi push notification terpusat 
	\item mempermudah proses pemantauan aplikasi.
\end{enumerate}

\section{Manfaat}
Dengan pengembangan yang dibuat dalam tugas akhir ini, diharapkan aplikasi push notification terpusat dapat menjadi platform pengiriman push notification yang dapat diandalkan di lingkungan Institut Teknologi Sepuluh Nopember. Aplikasi juga dirancang terdistribusi, sehingga memiliki skalabilitas yang lebih baik dari sebelumnya. Selain itu, dengan penambahan aplikasi monitoring diharapkan dapat mempermudah proses pemantauan aplikasi.

\section {Metodologi}
Metodologi pengerjaan yang digunakan pada tugas akhir ini memiliki beberapa tahapan. Tahapan-tahapan tersebut adalah sebagai berikut:

\begin{enumerate}

\item Penyusunan proposal\\
Pada tahapan ini, penulis akan menyusun rencana dan langkah-langkah yang akan dilakukan dalam proses pembuatan tugas akhir.

\item Studi literatur\\
Pada tahap ini, akan dicari studi literatur yang relevan untuk dijadikan referensi dalam pengerjaan tugas akhir, antara lain mengenai \textit{Java}, \textit{Spring}, \textit{Firebase Cloud Messaging}, \textit{Apple Push Notification Service}, dan \textit{Apache Kafka}.

\item Analisis dan Perancangan Sistem\\
Tahap ini meliputi perumusan kebutuhan fungsional, kebutuhan non-fungsional, kasus penggunaan, diagram aktivitas, diagram kelas, dan diagram sekuens.

\item Implementasi\\
Aplikasi ini diimplementasikan dengan menggunakan kakas bantu sebagai berikut:
\begin{enumerate}
\item Sistem basis data menggunakan Microsoft SQL Server.
\item Message queueing menggunakan Apache Kafka.
\item Aplikasi dibuat menggunakan bahasa pemrograman Java dengan kerangka kerja Spring.
\item Pembuatan aplikasi menggunakan IDE Intellij IDEA.
\end{enumerate}

\item Uji Coba dan Evaluasi\\
Pengujian dilakukan dengan mengirimkan push notification dalam jumlah besar untuk mengetahui tingkat keberhasilan dan kecepatan pengiriman notifikasi dengan model pengujian blackbox testing.

\item Penyusunan buku\\
Pada tahap ini dilakukan penyusunan laporan yang menjelaskan dasar teori dan metode yang digunakan dalam tugas akhir ini serta hasil dari implementasi aplikasi perangkat lunak yang telah dibuat.
\end{enumerate}

\section {Sistematika Penulisan}

Sistematika laporan tugas akhir yang akan digunakan adalah sebagai berikut:

\begin{enumerate}
\item Bab 1 : PENDAHULUAN

Bab ini menjelaskan konteks tugas akhir yang akan dikerjakan, termasuk latar belakang, rumusan masalah, batasan masalah, tujuan, manfaat, metodologi, dan sistematika penulisan.

\item Bab 2 : DASAR TEORI

Bab ini menjelaskan dasar teori yang akan digunakan dalam proses pengerjaan tugas akhir.

\item Bab 3 : ANALISIS DAN PERANCANGAN SISTEM

Bab ini menjelaskan tentang analisis permasalahan, deskripsi umum sistem, spesifikasi kebutuhan perangkat lunak, lingkungan perancangan, perancangan arsitektur sistem, dan diagram kelas berdasarkan dasar teori yang dijelaskan pada bab 2.

\item Bab 4 : IMPLEMENTASI SISTEM

Bab ini menjelaskan implementasi dari perancangan dan implementasi fitur-fitur penunjang aplikasi yang dijelaskan pada bab 3.

\item Bab 5 : PENGUJIAN DAN EVALUASI

Bab ini menjelaskan hasil pengujian dan evaluasi dengan metode \textit{blackbox} untuk mengetahui nilai fungsionalitas dari perangkat lunak yang telah diimplementasikan pada bab 4.

\item Bab 6 : PENUTUP

Bab ini berisi kesimpulan yang telah didapat dari hasil pengujian dan evaluasi yang telah dilakukan.
\end{enumerate}
