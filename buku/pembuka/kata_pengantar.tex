\chapter {KATA PENGANTAR}

Puji syukur penulis panjatkan kepada Tuhan Yang Maha Esa atas penyertaan dan karunia-Nya sehingga penulis dapat menyelesaikan tugas akhir dan laporan akhir dalam bentuk buku ini. Penelitian tugas akhir ini dilakukan untuk mengeksplorasi topik yang menarik perhatian penulis serta memenuhi salah satu syarat dalam mendapatkan gelar Sarjana Komputer di Departemen Informatika Fakultas Teknologi Informasi dan Komunikasi Institut Teknologi Sepuluh Nopember. Penulis memiliki harapan bahwa apa yang penulis kerjakan dapat membawa manfaat bagi perkembangan ilmu pengetahuan di bidang komputer dan bagi penulis sendiri selaku peneliti.

Penulis ingin mengucapkan terima kasih kepada semua pihak yang telah membimbing dan memberi dukungan baik secara langsung maupun tidak langsung selama proses pengerjaan tugas akhir ini maupun selama menempuh masa studi. Pihak tersebut antara lain:

\begin {enumerate}
	\item Keluarga besar penulis yang selalu memberikan doa dan semangat sehingga penulis dapat menyelesaikan tugas akhir dalam rentang waktu yang diharapkan.
	\item Bapak Rully Soelaiman S.Kom., M.Kom., selaku pembimbing penulis yang memberikan dukungan baik berupa didikan dan ajaran, maupun semangat dan nasihat selama menempuh masa studi dan pengerjaan tugas akhir. Berkat bimbingan dan ketersediaannya untuk berdiskusi, penulis dapat menyelesaikan tugas akhir dalam rentang waktu yang diharapkan.
	\item Ibu Wijayanti Nurul Khotimah, S.Kom., M.Sc., selaku pembimbing penulis yang memberikan dukungan berupa ilmu, arahan, dan semangat selama pengerjaan tugas akhir.
	\item Seluruh dosen dan karyawan Departemen Informatika Fakultas Teknologi Informasi dan Komunikasi Institut Teknologi Sepuluh Nopember yang telah memberi ilmu dan waktunya untuk mempersiapkan penulis agar siap untuk masuk ke dalam dunia kerja. 
	\item Teman-teman, kakak-kakak, dan adik-adik mahasiswa Departemen Informatika Fakultas Teknologi Informasi dan Komunikasi Institut Teknologi Sepuluh Nopember yang senantiasa membantu, menemani, memberi semangat dan kenangan selama kurang lebih 3,5 tahun masa studi.
	\item Keluarga \textit{administrator} Laboratorium Pemrograman 2 Departemen Informatika Fakultas Teknologi Informasi dan Komunikasi Institut Teknologi Sepuluh Nopember yang selalu menemani dan membantu penulis ketika ada kesusahan selama masa studi.
	\item Teman-teman kelompok GLBK yang sudah menemani, berjuang dan menorehkan banyak kenangan selama masa studi.
	\item Teman-teman kelompok Petualang yang sudah menemani penulis menelusuri berbagai hobi berpetualang selama masa studi.
\end {enumerate}

Penulis mohon maaf jika masih ada kekurangan pada tugas akhir ini. Penulis juga berharap tugas akhir ini dapat memberikan kontribusi dan manfaat bagi pembaca.

\begin{flushright}
Surabaya, Desember 2018 \\*
\vspace{5em}
\penulis
\end{flushright}