\chapter {ABSTRAK}

% ---- Indonesian vers.

\noindent\textbf{\MakeUppercase\judul}
\vspace*{1em}

\begin{tabularx}{\linewidth}{ l l X }
	Nama 			& : & \penulis \\
	NRP 			& :	& \nrp \\
	Departemen 		& : & \jurusan, \newline \fakultas, ITS \\
	Pembimbing I 	& : & \pembimbingsatu \\
	Pembimbing II 	& : & \pembimbingdua
	\vspace*{1em} 	% HACKY--USE ALTERNATIVE IF POSSIBLE %
\end {tabularx}

\noindent\textbf{Abstrak} \\
\itshape
\par \textit{Graceful labelling} dari suatu graf G dengan N \textit{node} adalah injeksi $ f : V(G) \rightarrow {0,1,2,...,N-1} $ sehingga ketika setiap \textit{edge} $ xy \in E(G) $ diberi label $ | f(x) - f(y) | $ maka semua label pada \textit{edge} berbeda. Hal ini pertama kali dicetuskan oleh Rosa pada tahun 1967.
\par Pada Tugas Akhir ini akan dirancang penyelesaian permasalahan \textit{graceful labelling} pada \textit{general tree} dengan batasan maksimal jumlah \textit{node} pada \textit{tree} adalah 27. Permasalahan ini dapat diselesaikan dengan menggunakan algoritma \textit{local search} dengan metode \textit{hill climbing} dan dibantu dengan batasan tabu. Hal yang harus perhatikan adalah bagaimana menghasilkan ruang pencarian solusi dari permasalahan ini sehingga dapat ditelusuri solusinya menggunakan teknik heuristik.
\par Hasil dari tugas akhir ini telah berhasil menyelesaikan permasalahan di atas dengan cukup efisien, dengan waktu penyelesaian terbaik 0.17 detik dan penggunaan memori 16 MB.


\vspace*{1em}
\noindent\bfseries Kata Kunci: graceful labelling; local search; hill climbing; heuristik;
\normalfont
\cleardoublepage

% ---- English vers.
\chapter {ABSTRACT}
\noindent\textbf{\MakeUppercase\juduleng}
\vspace*{1em}

\begin{tabularx}{\linewidth}{ l l X }
	Name 			& : & \penulis \\
	Student ID		& :	& \nrp \\
	Department 		& : & \jurusaneng, \newline \fakultaseng, ITS \\
	Supervisor I 	& : & \pembimbingsatu \\
	Supervisor II 	& : & \pembimbingdua
	\vspace*{1em} 	% HACKY--USE ALTERNATIVE IF POSSIBLE %
\end {tabularx}
	
\noindent\textbf{Abstract} \\
\itshape
\par A graceful labelling of a graph G with N nodes is an injection $ f : V(G) \rightarrow {0,1,2,...,N-1} $ such that when each edge $ xy \in E(G) $ is assigned the label, $ | f(x) - f(y) | $, all of the edge labels are distinct. This idea was introduced by Rosa in 1967.
\par This undergraduate thesis will design the problem solving of graceful labelling in general tree with nodes in tree limited to 27. This problem can be solved using local search algorithm with hill climbing method and tabu limit. The difficult thing from this method is the correct way to generate the solution search space so it can be traversed by heuristic method.
\par The result shows that graceful labelling problem in general tree is succesfully solved efficiently with best time of 0.17 seconds and memory usage of 16 MB.

\vspace*{1em}
\noindent\bfseries Keywords: graceful labelling; local search; hill climbing; heuristic;
\normalfont
\cleardoublepage