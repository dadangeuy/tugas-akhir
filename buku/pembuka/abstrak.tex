% ---- Indonesian vers.
\begin{center}
	\centering\noindent\textbf{\MakeUppercase{Pengembangan Antrian Pesan pada Aplikasi Push Notification Terpusat dengan Apache Kafka}}
\end{center}
\vspace*{1em}

\noindent\begin{tabularx}{\linewidth}{l l X}
	Nama & : & \penulis \\
	NRP & :	& \nrp \\
	Departemen & : & \jurusan, \newline \fakultas, ITS \\
	Pembimbing I & : & \pembimbingsatu \\
	Pembimbing II & : & \pembimbingdua
\end {tabularx}

{\let\clearpage\relax\titlespacing{\chapter}{0em}{0em}{1em} \chapter{ABSTRAK}}
\itshape
\par Push notification adalah pesan yang muncul di perangkat pengguna lewat aplikasi yang terpasang dalam perangkat tersebut. Push notification pada umumnya digunakan untuk menyebarkan informasi seperti promo, pengingat pembayaran, berita terbaru, dan sebagainya.
\par Pengiriman push notification di Institut Teknologi Sepuluh Nopember menggunakan aplikasi Push Notification Terpusat. Push Notification Terpusat adalah layanan pengiriman push notification berbasis web yang dikembangkan untuk memudahkan manajemen dan penyebaran informasi ke pengguna aplikasi yang ada di Institut Teknologi Sepuluh Nopember.
\par Pengiriman push notification dilakukan dengan cara mengirim request HTTP ke layanan Apple Push Notification Service dan Firebase Cloud Messaging. Karena jumlah push notification yang dikirim bisa sangat banyak, Push Notification Terpusat menggunakan arsitektur antrian pesan untuk mencegah sistem kehabisan sumber daya.
\par Berdasarkan hasil uji, antrian pesan yang digunakan Push Notification Terpusat hanya mampu menangani sekitar 1.900 push notification. Pada tugas akhir ini, Push Notification Terpusat akan dikembangkan dengan menggunakan Kafka sebagai antrian pesan. Aplikasi akan diimplementasikan dengan bahasa pemrograman Java dan kerangka kerja Spring.
\par Pengujian pada tugas akhir ini meliputi pengujian fungsional dan non fungsional dengan menggunakan metode blackbox. Dari hasil uji, sistem yang dibuat dapat memenuhi kebutuhan fungsional yang ada di aplikasi sebelumnya, dengan peningkatan kecepatan hingga 138 kali lebih cepat, dan mampu menangani hingga 199.980 push notification dalam satu waktu.

\vspace*{1em}
\noindent\bfseries Kata Kunci: Push Notification, Apple Push Notification Service, Firebase Cloud Messaging, Apache Kafka
\normalfont
\cleardoublepage

% ---- English vers.
\begin{center}
	\noindent\textbf{\MakeUppercase{Message Queue Development in Centralized Push Notification Application with Apache Kafka}}
\end{center}
\vspace*{1em}

\noindent\begin{tabularx}{\linewidth}{l l X}
	Name 			& : & \penulis \\
	Student ID		& :	& \nrp \\
	Department 		& : & \jurusaneng, \newline \fakultaseng, ITS \\
	Supervisor I 	& : & \pembimbingsatu \\
	Supervisor II 	& : & \pembimbingdua
\end {tabularx}

{\let\clearpage\relax\titlespacing{\chapter}{0em}{0em}{1em} \chapter{ABSTRACT}}
\itshape
\par Push notification is a message that shown in user device using an installed application inside the device. Push notification usually is used to distribute information like promo, payment reminder, new news, et cetera.
\par Institut Teknologi Sepuluh Nopember use Centralized Push Notification application to send push notification. Centralized Push Notification is a push notification management service based on web platform, developed to simplify management and distributing information to user of application in Institut Teknologi Sepuluh Nopember.
\par Push notification delivery is done by sending an HTTP request to Apple Push Notification Service and Firebase Cloud Messaging. Because of the amount of push notification sent can be very large, Centralized Push Notification use message queue architecture to prevent the system from running out of resource.
\par Based on the test result, the message queue used by Centralized Push Notification is only able to handle around 1.900 push notification. In this undergraduate thesis, writer will improve Centralized Push Notification using Kafka as message queue. The application will be implemented in Java as programming language and Spring as framework.
\par Testing in this undergraduate thesis include functional and non functional test using blackbox method. From the test result, the new system can fulfill all functional requirement from the old system, with delivery speed increasing up to 138 times faster, and able to handle up to 199.980 push notification at the same time.

\vspace*{1em}
\noindent\bfseries Keywords: Push Notification, Apple Push Notification Service, Firebase Cloud Messaging, Apache Kafka
\normalfont
\cleardoublepage