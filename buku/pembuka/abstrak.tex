\chapter {ABSTRAK}

% ---- Indonesian vers.

\noindent\textbf{\MakeUppercase\judul}
\vspace*{1em}

\begin{tabularx}{\linewidth}{ l l X }
	Nama 			& : & \penulis \\
	NRP 			& :	& \nrp \\
	Departemen 		& : & \jurusan, \newline \fakultas, ITS \\
	Pembimbing I 	& : & \pembimbingsatu \\
	Pembimbing II 	& : & \pembimbingdua
	\vspace*{1em} 	% HACKY--USE ALTERNATIVE IF POSSIBLE %
\end {tabularx}

\noindent\textbf{Abstrak} \\
\itshape
\par Aplikasi \textit{push notification} terpusat merupakan layanan pengiriman push notification berbasis web. Aplikasi ini dikembangkan dengan tujuan untuk mempermudah manajemen pengiriman push notification ke perangkat Android, Web dan iOS di lingkup Institut Teknologi Sepuluh Nopember.
\par Pada tugas akhir ini, penulis mengembangkan aplikasi \textit{push notification} terpusat dengan cara mengganti implementasi message queue yang sudah ada dengan Kafka. Aplikasi ini akan dikembangkan menggunakan bahasa pemrograman Java dengan kerangka kerja Spring, message queue Kafka, dan sistem basis data SQL Server. Untuk mengimplementasikan aplikasi ini, penulis membagi aplikasi menjadi tiga modul, yaitu \textit{Scheduler}, \textit{Sender APN} dan \textit{Sender FCN}. \textit{Scheduler} bertanggung jawab untuk penjadwalan pengiriman notifikasi. Sender APN bertanggung jawab untuk mengirimkan notifikasi ke perangkat iOS lewat layanan Apple Push Notification Service (APNs). Sedangkan Sender FCM bertanggung jawab untuk mengirimkan notifikasi ke perangkat Android dan Web lewat layanan Firebase Cloud Messaging (FCM).

\vspace*{1em}
\noindent\bfseries Kata Kunci: Push Notification, Apple Push Notification Service, Firebase Cloud Messaging, Apache Kafka
\normalfont
\cleardoublepage

% ---- English vers.
\chapter {ABSTRACT}
\noindent\textbf{\MakeUppercase\juduleng}
\vspace*{1em}

\begin{tabularx}{\linewidth}{ l l X }
	Name 			& : & \penulis \\
	Student ID		& :	& \nrp \\
	Department 		& : & \jurusaneng, \newline \fakultaseng, ITS \\
	Supervisor I 	& : & \pembimbingsatu \\
	Supervisor II 	& : & \pembimbingdua
	\vspace*{1em} 	% HACKY--USE ALTERNATIVE IF POSSIBLE %
\end {tabularx}
	
\noindent\textbf{Abstract} \\
\itshape
\par 

\vspace*{1em}
\noindent\bfseries Keywords: Push Notification, Apple Push Notification Service, Firebase Cloud Messaging, Apache Kafka
\normalfont
\cleardoublepage