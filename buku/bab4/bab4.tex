\chapter {IMPLEMENTASI}
\par Bab ini membahas implementasi dari rancangan sistem yang ditulis dengan menggunakan bahasa pemrograman Java serta konfigurasi dengan bahasa YAML.

\section{Lingkungan Implementasi}
\par Lingkungan implementasi yang digunakan untuk mengembangkan tugas akhir ini memiliki spesifikasi perangkat keras dan lunak seperti ditampilkan pada Tabel \ref{tabel_spesifikasi_server}, Tabel \ref{tabel_spesifikasi_perangkat_android}, dan Tabel \ref{tabel_spesifikasi_perangkat_ios}.
\begin{longtable}{|p{3cm}|p{6.5cm}|}
	\caption{Spesifikasi Server Push Notification Terpusat, Kafka, Zookeeper, dan SQL Server} \label{tabel_spesifikasi_server} \\ \hline
	\rowcolor{lightgray} Komponen & Spesifikasi \\ \hline
	\endfirsthead
	\hline
	\rowcolor{lightgray} Komponen & Spesifikasi \\ \hline
	\endhead
	CPU & AMD Ryzen 5 2500U \\ \hline
	CPU Core & 4 \\ \hline
	Memory & 12 GB \\ \hline
	Sistem Operasi & Ubuntu 19.04 \\ \hline
	IDE & Intellij IDEA \\ \hline
	Aplikasi & \begin{enumerate}
		\item Kafka versi 2.3.0
		\item Zookeeper versi 3.5.5
		\item SQLServer versi 2012
	\end{enumerate} \\ \hline
\end{longtable}
\begin{longtable}{|p{3cm}|p{6.5cm}|}
	\caption{Spesifikasi Perangkat Android} \label{tabel_spesifikasi_perangkat_android} \\ \hline
	\rowcolor{lightgray} Komponen & Spesifikasi \\ \hline
	\endfirsthead
	\hline
	\rowcolor{lightgray} Komponen & Spesifikasi \\ \hline
	\endhead
	CPU & Snapdragon 636 \\ \hline
	Memory & 3 GB \\ \hline
	Sistem Operasi & Android 9 (Pie) \\ \hline
	Aplikasi & Push Notification Dev versi 1.0 \\ \hline
\end{longtable}
\begin{longtable}{|p{3cm}|p{6.5cm}|}
	\caption{Spesifikasi Perangkat iOS} \label{tabel_spesifikasi_perangkat_ios} \\ \hline
	\rowcolor{lightgray} Komponen & Spesifikasi \\ \hline
	\endfirsthead
	\hline
	\rowcolor{lightgray} Komponen & Spesifikasi \\ \hline
	\endhead
	CPU & Apple A10 Fusion \\ \hline
	Memory & 3 GB \\ \hline
	Sistem Operasi & iOS 12 \\ \hline
	Aplikasi & MyITS Wali versi 1.0.2 \\ \hline
\end{longtable}

\section{Implementasi Basis Data}
\par Subbab ini membahas struktur tabel yang digunakan, meliputi tujuan pembuatan tabel, Kode yang digunakan untuk membuat tabel, dan gambar tabel yang telah dibuat.

%\subsection{Implementasi Tabel User}
%\par Tabel user digunakan untuk menyimpan data pengguna. Kode yang digunakan untuk membuat tabel dapat dilihat pada Kode Sumber \ref{code:user_account} dan hasil tabel dapat dilihat pada Gambar \ref{tabel_user_account}.
%\lstinputlisting[label=code:user_account, caption=Implementasi Tabel User, language=SQL]{bab4/sql/user_account.sql}
%\clearpage
%\begin{figure}[H]
%    \centering\includegraphics[width=0.5\textwidth]{bab4/figures/tabel_user_account.jpg}
%    \caption{Tabel User (user\_account)}
%    \label{tabel_user_account}
%\end{figure}
%\clearpage

\subsection{Implementasi Tabel Group}
\par Tabel \textbf{group} digunakan untuk menyimpan data kelompok pengguna. Kode yang digunakan untuk membuat tabel dapat dilihat pada Kode Sumber \ref{code:pn_group} dan hasil tabel dapat dilihat pada Gambar \ref{tabel_pn_group}.
\lstinputlisting[label=code:pn_group, caption=Implementasi Tabel Group, language=SQL]{bab4/sql/pn_group.sql}
\begin{figure}[H]
    \centering\includegraphics[width=0.45\textwidth]{bab4/figures/tabel_pn_group.jpg}
    \caption{Tabel Group}
    \label{tabel_pn_group}
\end{figure}

\subsection{Implementasi Tabel Group Member}
\par Tabel \textbf{group member} digunakan untuk menyimpan data anggota kelompok pengguna. Kode yang digunakan untuk membuat tabel dapat dilihat pada Kode Sumber \ref{code:pn_group_member} dan hasil tabel dapat dilihat pada Gambar \ref{tabel_pn_group_member}.
\lstinputlisting[label=code:pn_group_member, caption=Implementasi Tabel Group Member, language=SQL]{bab4/sql/pn_group_member.sql}
\clearpage
\begin{figure}[H]
    \centering\includegraphics[width=0.45\textwidth]{bab4/figures/tabel_pn_group_member.jpg}
    \caption{Tabel Group Member}
    \label{tabel_pn_group_member}
\end{figure}

%\subsection{Implementasi Tabel Client}
%\par Tabel client digunakan untuk menyimpan data aplikasi. Kode yang digunakan untuk membuat tabel dapat dilihat pada Kode Sumber \ref{code:oauth_client} dan hasil tabel dapat dilihat pada Gambar \ref{tabel_oauth_client}.
%\lstinputlisting[label=code:oauth_client, caption=Implementasi Tabel Client, language=SQL]{bab4/sql/oauth_client.sql}
%\begin{figure}[H]
%    \centering\includegraphics[width=0.5\textwidth]{bab4/figures/tabel_oauth_client.jpg}
%    \caption{Tabel Client (oauth\_client)}
%    \label{tabel_oauth_client}
%\end{figure}

\subsection{Implementasi Tabel Certificate}
\par Tabel \textbf{certificate} digunakan untuk menyimpan data sertifikat \textit{client} untuk autentikasi ke layanan APNs dan FCM. Kode yang digunakan untuk membuat tabel dapat dilihat pada Kode Sumber \ref{code:pn_certificate} dan hasil tabel dapat dilihat pada Gambar \ref{tabel_pn_certificate}.
\lstinputlisting[label=code:pn_certificate, caption=Implementasi Tabel Certificate, language=SQL]{bab4/sql/pn_certificate.sql}
\clearpage
\begin{figure}[H]
    \centering\includegraphics[width=0.45\textwidth]{bab4/figures/tabel_pn_certificate.jpg}
    \caption{Tabel Certificate}
    \label{tabel_pn_certificate}
\end{figure}

\subsection{Implementasi Tabel Device}
\par Tabel \textbf{device} digunakan untuk menyimpan data perangkat pengguna yang terdaftar di layanan APNs dan FCM. Kode yang digunakan untuk membuat tabel dapat dilihat pada Kode Sumber \ref{code:device_token} dan hasil tabel dapat dilihat pada Gambar \ref{tabel_device_token}.
\lstinputlisting[label=code:device_token, caption=Implementasi Tabel Device, language=SQL]{bab4/sql/device_token.sql}
\begin{figure}[H]
    \centering\includegraphics[width=0.45\textwidth]{bab4/figures/tabel_device_token.jpg}
    \caption{Tabel Device}
    \label{tabel_device_token}
\end{figure}

\subsection{Implementasi Tabel Batch}
\par Tabel \textbf{batch} digunakan untuk menyimpan data notifikasi yang akan dikirim ke beberapa pengguna atau kelompok pengguna. Kode yang digunakan untuk membuat tabel dapat dilihat pada Kode Sumber \ref{code:pn_batch} dan hasil tabel dapat dilihat pada Gambar \ref{tabel_pn_batch}.
\lstinputlisting[label=code:pn_batch, caption=Implementasi Tabel Batch, language=SQL]{bab4/sql/pn_batch.sql}
\begin{figure}[H]
    \centering\includegraphics[width=0.5\textwidth]{bab4/figures/tabel_pn_batch.jpg}
    \caption{Tabel Batch}
    \label{tabel_pn_batch}
\end{figure}

\subsection{Implementasi Tabel Packet}
\par Tabel \textbf{packet} digunakan untuk menyimpan data notifikasi yang akan dikirim ke satu perangkat. Kode yang digunakan untuk membuat tabel dapat dilihat pada Kode Sumber \ref{code:pn_packet} dan hasil tabel dapat dilihat pada Gambar \ref{tabel_pn_packet}.
\lstinputlisting[label=code:pn_packet, caption=Implementasi Tabel Packet, language=SQL]{bab4/sql/pn_packet.sql}
\begin{figure}[H]
    \centering\includegraphics[width=0.5\textwidth]{bab4/figures/tabel_pn_packet.jpg}
    \caption{Tabel Packet}
    \label{tabel_pn_packet}
\end{figure}

\subsection{Implementasi Tabel User Destination}
\par Tabel \textbf{user destination} digunakan untuk menyimpan data pengguna yang akan menerima notifikasi dari suatu \textit{batch}. Kode yang digunakan untuk membuat tabel dapat dilihat pada Kode Sumber \ref{code:pn_user_destination} dan hasil tabel dapat dilihat pada Gambar \ref{tabel_pn_user_destination}.
\lstinputlisting[label=code:pn_user_destination, caption=Implementasi Tabel User Destination, language=SQL]{bab4/sql/pn_user_destination.sql}
\begin{figure}[H]
    \centering\includegraphics[width=0.5\textwidth]{bab4/figures/tabel_pn_user_destination.jpg}
    \caption{Tabel User Destination}
    \label{tabel_pn_user_destination}
\end{figure}

\subsection{Implementasi Tabel Group Destination}
\par Tabel \textbf{group destination} digunakan untuk menyimpan data kelompok yang akan menerima notifikasi dari suatu \textit{batch}. Kode yang digunakan untuk membuat tabel dapat dilihat pada Kode Sumber \ref{code:pn_group_destination} dan hasil tabel dapat dilihat pada Gambar \ref{tabel_pn_group_destination}.
\lstinputlisting[label=code:pn_group_destination, caption=Implementasi Tabel Group Destination, language=SQL]{bab4/sql/pn_group_destination.sql}
\begin{figure}[H]
    \centering\includegraphics[width=0.5\textwidth]{bab4/figures/tabel_pn_group_destination.jpg}
    \caption{Tabel Group Destination}
    \label{tabel_pn_group_destination}
\end{figure}

\section{Implementasi Antrian Pesan}
\par Subbab ini membahas bagaimana mengimplementasikan antrian pesan dengan menggunakan Kafka, meliputi konfigurasi Kafka dan sistem lain yang dibutuhkan oleh Kafka.

\subsection{Implementasi Zookeeper}
\par Zookeeper dijalankan sebagai Docker Container dengan konfigurasi yang diatur oleh Docker Compose. Zookeeper dibutuhkan oleh Kafka untuk mengatur distribusi topik, \textit{consumer}, dan sebagainya. Untuk keamanan, Zookeeper dilindungi dengan metode autentikasi SASL/Digest-MD5. Hasil implementasi dapat dilihat pada Kode Sumber \ref{yaml:zookeeper}.
\lstinputlisting[label=yaml:zookeeper, caption=Konfigurasi Docker Compose untuk Zookeeper] {bab4/yaml/zookeeper.yml}

\subsection{Implementasi Kafka}
\par Kafka akan berjalan sebagai Docker Container dengan konfigurasi yang diatur oleh Docker Compose. Untuk meningkatkan performa, partisi topik ditambah menjadi 4 dan jumlah \textit{thread} jaringan ditambah menjadi 16. Untuk keamanan, Kafka dilindungi dengan metode autentikasi SASL/PLAINTEXT. Hasil implementasi dapat dilihat pada Kode Sumber \ref{yaml:kafka}.
\lstinputlisting[label=yaml:kafka, caption=Konfigurasi Docker Compose untuk Kafka] {bab4/yaml/kafka.yml}

\section{Implementasi Proses dan Kasus Penggunaan}
\par Subbab ini membahas implementasi proses yang dibuat berdasarkan hasil analisa dan rancangan yang telah dilakukan.

\subsection{Implementasi Pembuatan Packet}
\par Proses pembuatan \textit{packet} dilakukan secara berkala oleh modul Scheduler setiap 30 detik. \textit{Packet} yang telah dibuat akan disimpan di sistem basis data. Hasil implementasi dapat dilihat pada Kode Sumber \ref{code:pembuatan_packet}.
\lstinputlisting[label=code:pembuatan_packet, caption=Implementasi Java untuk Pembuatan \textit{Packet}, language=Java] {bab4/java/pembuatan_packet.java}

\subsection{Implementasi Menambahkan Packet ke Antrian}
\par Proses menambahkan \textit{packet} ke antrian dilakukan secara berkala oleh modul Scheduler setiap 30 detik dengan jeda awal 15 detik. Tujuan penambahan jeda awal adalah untuk mengoptimalkan penggunaan sumber daya \textit{server} dengan mencegah proses pembuatan dan mengantrikan \textit{packet} berjalan bersamaan. \textit{Packet} yang siap dikirim akan diantrikan ke Kafka. Hasil implementasi dapat dilihat pada Kode Sumber \ref{code:menambahkan_packet_ke_antrian}.
\lstinputlisting[label=code:menambahkan_packet_ke_antrian, caption=Implementasi Java untuk Menambahkan \textit{Packet} ke Antrian, language=Java] {bab4/java/menambahkan_packet_ke_antrian.java}

\subsection{Implementasi Pengiriman Packet ke APNs}
\par Proses pengiriman \textit{packet} ke APNs dilakukan secara berkala oleh Sender APN dengan cara menunggu Kafka untuk memberikan \textit{packet} yang berada diantrian topik "ios". Hasil implementasi dapat dilihat pada Kode Sumber \ref{code:pengiriman_packet_ke_apns}.
\lstinputlisting[label=code:pengiriman_packet_ke_apns, caption=Implementasi Java untuk Pengiriman \textit{Packet} ke APNs, language=Java] {bab4/java/pengiriman_packet_ke_apns.java}

\subsection{Implementasi Pengiriman Packet ke FCM}
\par Proses pengiriman \textit{packet} dilakukan secara berkala oleh Sender FCM dengan cara menunggu Kafka untuk mengirimkan data \textit{packet} yang berada diantrian topik "android" dan "web". Hasil implementasi dapat dilihat pada Kode Sumber \ref{code:pengiriman_packet_ke_fcm}.
\lstinputlisting[label=code:pengiriman_packet_ke_fcm, caption=Implementasi Java untuk Pengiriman \textit{Packet} ke FCM, language=Java] {bab4/java/pengiriman_packet_ke_fcm.java}

\subsection{Implementasi Pemantauan Aplikasi}
\par Proses pemantauan aplikasi (menampilkan penggunaan sumber daya, menampilkan status kesehatan, menampilkan konfigurasi, menampilkan \textit{log}) diimplementasikan dengan menggunakan pustaka Actuator yang sudah terintegrasi dengan kerangka kerja Spring. Untuk keamanan, API Actuator dilindungi dengan metode autentikasi HTTP Basic. Setiap modul aplikasi (Scheduler, Sender APN, dan Sender FCM) menggunakan konfigurasi Actuator yang sama. Hasil implementasi konfigurasi Actuator dapat dilihat pada Kode Sumber \ref{yaml:actuator}.
\lstinputlisting[label=yaml:actuator, caption=Implementasi YAML untuk Konfigurasi Actuator] {bab4/yaml/actuator.yml}
