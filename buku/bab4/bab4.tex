\chapter {IMPLEMENTASI}

\section{Lingkungan Implementasi}

Lingkungan implementasi dan pengembangan menggunakan sebuah komputer dengan spesifikasi perangkat lunak dan perangkat keras sebagai berikut.

\begin{enumerate}
	\item Perangkat Keras
	\begin{enumerate}
		\item Processor Intel® Core™ i5-3320M CPU @ 2.60GHz
		\item Random Access Memory 4GB
	\end{enumerate}
	\item Perangkat Lunak
	\begin{enumerate}
		\item Sistem Operasi Manjaro Illyria 18.0 64-bit
		\item Text Editor Visual Studio Code 1.29.1
		\item Bahasa Pemrograman C++
		\item g++ (GCC) 8.2.1 20181127
	\end{enumerate}
\end{enumerate}

\section{Implementasi Kelas Node}
Subbab ini menjabarkan implementasi dari kelas \textit{Node} yang dijelaskan pada subbab \ref{desain_kelas_node}. Implementasi kelas \textit{Node} dapat dilihat pada Kode Sumber \ref{code:class_node}.

\begin{lstlisting}[language=C++, caption=Implementasi Kelas \textit{Node}, label=code:class_node, firstnumber=0]
	class Node
    public:
        int value;
        vector<int> neighbour;
\end{lstlisting}

\section{Implementasi Kelas Tree}
Subbab ini menjabarkan implementasi dari kelas \textit{Tree} yang dijelaskan pada subbab \ref{desain_kelas_tree}. Kelas ini menyimpan berbagai data dan operasi yang digunakan dalam sistem. Implementasi kelas \textit{Tree} dapat dilihat pada Kode Sumber \ref{code:class_tree}.

\begin{lstlisting}[language=C++, firstnumber=0]
	class Tree
	{
	  private:
		int edgeValuesCount[30];
		int distinctEdgeValue;
		int numberOfNode;
		vector<Node *> *nodes;
	
		void incrementEdgeValue(int value)
		{
			if (!this->edgeValuesCount[value]) {
				++this->distinctEdgeValue;
			}
			++this->edgeValuesCount[value];
		}
	
		void decrementEdgeValue(int value)
		{
			this->edgeValuesCount[value]--;
			if (!this->edgeValuesCount[value])
			{
				this->distinctEdgeValue--;
				return;
			}
		}
	
		Node *getNode(int id)
		{
			return this->nodes->at(id);
		}
	
	\end{lstlisting}
	
	\pagebreak
	\begin{lstlisting}[language=C++, firstnumber=32]
		void populateEdgeValueFromNode(Node *node, int ignore)
		{
			for (vector<int>::iterator it = node->neighbour.begin(); it != node->neighbour.end(); ++it)
			{
				if (*it == ignore)
					continue;
				Node *otherNode = this->getNode(*it);
				this->incrementEdgeValue(getEdgeValue(node, otherNode));
			}
		}
	
		void clearEdgeValueFromNode(Node *node, int ignore)
		{
			for (vector<int>::iterator it = node->neighbour.begin(); it != node->neighbour.end(); ++it)
			{
				if (*it == ignore)
					continue;
				Node *otherNode = this->getNode(*it);
				this->decrementEdgeValue(getEdgeValue(node, otherNode));
			}
		}
	
	  public:
		Tree(int numberOfNodeInp){
			this->numberOfNode = numberOfNodeInp;
			memset(this->edgeValuesCount, 0, sizeof(this->edgeValuesCount));
			distinctEdgeValue = 0;
			this->nodes = new vector<Node *>(numberOfNodeInp, nullptr);
			for (int i = 0; i < numberOfNodeInp; ++i)
		\end{lstlisting}
		\pagebreak
		\begin{lstlisting}[language=C++, firstnumber=62]
			{
				Node *newNode = new Node();
				newNode->value = i;
				newNode->neighbour = vector<int>();
				nodes->at(i) = newNode;
			}
		}
	
		vector<Node*>* getNodes(){
			return this->nodes;
		}
	
		int getNumberOfNode(){
			return this->numberOfNode;
		}
	
		void connectNode(int a, int b)
		{
			Node *nodeA, *nodeB;
			nodeA = this->getNode(a);
			nodeB = this->getNode(b);
			nodeA->neighbour.push_back(b);
			nodeB->neighbour.push_back(a);
			int edgeValue = getEdgeValue(nodeA, nodeB);
			this->incrementEdgeValue(edgeValue);
		}
	
		int countDistinctEdgeValue()
		{
			return this->distinctEdgeValue;
		}
	
		void swapNodeValue(int a, int b)
		{
			Node *nodeA, *nodeB;
			nodeA = this->nodes->at(a);
			nodeB = this->nodes->at(b);
			this->clearEdgeValueFromNode(nodeA, -1);
			this->clearEdgeValueFromNode(nodeB, a);
		\end{lstlisting}
		\pagebreak
		\begin{lstlisting}[language=C++, caption=Implementasi Kelas \textit{Tree}, label=code:class_tree, firstnumber=102]
			int temp = nodeA->value;
			nodeA->value = nodeB->value;
			nodeB->value = temp;
			this->populateEdgeValueFromNode(nodeA, -1);
			this->populateEdgeValueFromNode(nodeB, a);
			return;
		}
	
		void print(){
			for (int i = 0; i < numberOfNode; ++i)
			{
				if (i)
					putchar(' ');
				printf("%d", nodes->at(i)->value);
			}
			putchar('\n');
		}
	};
\end{lstlisting}

\section{Implementasi Kelas Tabu}
Subbab ini menjabarkan implementasi dari kelas batasan tabu yang dijelaskan pada subbab \ref{desain_kelas_tabu}. Kelas ini berfungsi untuk mencatat data langkah tabu dan aksi yang dibatasi oleh tabu. Implementasi kelas batasan tabu dapat dilihat pada Kode Sumber \ref{code:class_tabu}.

\begin{lstlisting}[language=C++, firstnumber=0]
	class TabuLimit
    private:
        int stepCounter;
        int limit;
        int choosenAt[2800];

\end{lstlisting}
\pagebreak
\begin{lstlisting}[language=C++, caption=Implementasi Kelas Batasan Tabu, label=code:class_tabu, firstnumber=6]
    public:
        TabuLimit(int limit): limit(limit), stepCounter(limit + 1)
        {
            memset(choosenAt, 0, sizeof(choosenAt));
        }

        void choose(int actionId) {
            choosenAt[actionId] = stepCounter;
            ++stepCounter;
        }

        bool isAllowed(int actionId) {
            return (stepCounter - choosenAt[actionId]) > limit;
        }
\end{lstlisting}

\section{Implementasi Fungsi Utama}
Subbab ini menjabarkan implementasi dari fungsi utama yang dijelaskan pada subbab \ref{desain_fungsi_solusi}. Implementasi fungsi ini dapat dilihat pada Kode Sumber \ref{code:solution}.

\begin{lstlisting}[language=C++, caption=Implementasi Fungsi Utama, label=code:solution, firstnumber=0]
	void solution()
	int numberOfNode;
	scanf("%d", &numberOfNode);
	Tree tree(numberOfNode);
	inputTree(&tree);
	searchAnswer(&tree);
	tree.print();
	return;
\end{lstlisting}

\section{Implementasi Fungsi Masukan Tree}
Subbab ini menjabarkan implementasi dari fungsi masukan \textit{tree} yang dijelaskan pada subbab \ref{desain_fungsi_masukan_tree}. Implementasi fungsi ini dapat dilihat pada Kode Sumber \ref{code:inputTree}.

\begin{lstlisting}[language=C++, caption=Implementasi Fungsi Masukan \textit{Tree}, label=code:inputTree, firstnumber=0]
	void inputTree(Tree *tree)
    for (int i = 1; i < tree->getNumberOfNode(); ++i)
    {
        int u, v;
        scanf("%d%d", &u, &v);
        --u;
        --v;
        tree->connectNode(u, v);
    }
\end{lstlisting}

\section{Implementasi Fungsi Pencarian Jawaban}
Subbab ini menjabarkan implementasi dari fungsi pencarian jawaban yang dijelaskan pada subbab \ref{desain_fungsi_pencarian_jawaban}.  Implementasi fungsi ini dapat dilihat pada Kode Sumber \ref{code:searchAnswer}.

\begin{lstlisting}[language=C++, firstnumber=0]
	void searchAnswer(Tree *tree)
    TabuLimit tabu(15);
    int currentDistinctEdgeValue = tree->countDistinctEdgeValue();
    while (currentDistinctEdgeValue < tree->getNumberOfNode() - 1)
    {
        pair<int, int> possibleSwap;
        int maxDistinctEdgeValue = 0;
		bool foundBetterConfiguration = false;
		\end{lstlisting}
		\pagebreak
		\begin{lstlisting}[language=C++, firstnumber=8]
        for (int i = 0; i < tree->getNumberOfNode(); ++i)
        {
            for (int j = i + 1; j < tree->getNumberOfNode(); ++j)
            {
                int actionId = (i * 100) + j;
                if (!tabu.isAllowed(actionId))
                {
                    continue;
                }
                tree->swapNodeValue(i, j);
                int possibleDistinctEdgeValue = tree->countDistinctEdgeValue();
                if (possibleDistinctEdgeValue > currentDistinctEdgeValue)
                {
                    foundBetterConfiguration = true;
                    currentDistinctEdgeValue = possibleDistinctEdgeValue;
                    tabu.choose(actionId);
                    continue;
				}
                if (
                    (!foundBetterConfiguration) &&
                    possibleDistinctEdgeValue > maxDistinctEdgeValue)
                {
                    maxDistinctEdgeValue = possibleDistinctEdgeValue;
                    possibleSwap = make_pair(i, j);
                }
                tree->swapNodeValue(i, j);
            }
		}
	\end{lstlisting}
	\pagebreak
	\begin{lstlisting}[language=C++, caption=Implementasi Fungsi Masukan \textit{Tree}, label=code:searchAnswer, firstnumber=36]
        if (!foundBetterConfiguration)
        {
            pair<int, int> choosenSwap = possibleSwap;
            tree->swapNodeValue(choosenSwap.first, choosenSwap.second);
            currentDistinctEdgeValue = tree->countDistinctEdgeValue();
            tabu.choose((choosenSwap.first * 100) + choosenSwap.second);
        }
    }
\end{lstlisting}