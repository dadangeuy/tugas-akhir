\chapter{KESIMPULAN DAN SARAN}
Pada bab ini akan diberikan kesimpulan yang diambil selama pengerjaan Tugas Akhir serta saran-saran tentang pengembangan yang dapat dilakukan terhadap Tugas Akhir ini di masa yang akan datang.

\section{Kesimpulan}
Berdasarkan penjabaran di bab-bab sebelumnya, dapat disimpulkan beberapa poin terkait penyelesaian permasalahan \problem.
\begin{enumerate}
	\item Sistem yang diajukan dapat menyelesaikan permasalahan \textit{graceful labelling} pada \textit{general tree} dengan benar dan cepat.
	\item Permasalahan \textit{graceful labelling} pada \textit{general tree} dapat diselesaikan dengan metode heuristik menggunakan teknik \textit{local search}.
	\item Sistem yang diajukan paling cepat membutuhkan waktu 0.17 detik dan memori 16 MB pada situs penilaian daring SPOJ untuk permasalahan \problem.
	\item Sistem yang diajukan memiliki kinerja yang lebih baik dari sistem yang menggunakan algoritma naif untuk menyelesaikan permasalahan \textit{graceful labelling} pada \textit{general tree}.
\end{enumerate}

\section{Saran}
Berikut merupakan beberapa saran untuk pengembangan sistem di masa yang akan datang. Saran-saran ini didasarkan pada hasil perancangan, implementasi, dan uji coba yang telah dilakukan.
\begin{enumerate}
	\item Desain dan implementasi sistem dapat dikembangkan lebih efisien lagi terutama bagian pencarian tetangga yang memiliki nilai yang lebih baik karena sekarang masih mencoba semua kemungkinan yang ada.
	\item Pemilihan label awal dapat dilakukan lebih baik lagi untuk menghasilkan titik mulai yang lebih baik.
	\item Metode pencarian ruang solusi pada Tugas Akhir ini dapat digunakan untuk permasalahan lain yang belum bisa diselesaikan.
\end{enumerate}