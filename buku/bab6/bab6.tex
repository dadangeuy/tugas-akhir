\chapter{KESIMPULAN DAN SARAN}
\par Bab ini membahas kesimpulan yang diperoleh selama pengerjaan tugas akhir dan saran yang dapat dilakukan untuk mengembangkan tugas akhir ini.

\section{Kesimpulan}
\begin{enumerate}
	\item Implementasi antrian pesan dengan Kafka dapat dilakukan dengan menggunakan Docker untuk menjalankan Kafka dan Zookeeper, lalu menggunakan Spring untuk mengirim dan mengambil pesan dari antrian.
	\item Implementasi penjadwalan pengiriman \textit{push notification} dapat dilakukan dengan menyimpan data \textit{push notification} ke sistem basis data, lalu membuat layanan penjadwalan yang mengambil dan mengirim \textit{push notification} dari sistem basis data secara periodik.
	\item Implementasi pengiriman \textit{push notification} ke perangkat Android dapat dilakukan dengan mengirim \textit{request push notification} ke layanan FCM, sedangkan pengiriman ke perangkat iOS dapat dilakukan dengan mengirim \textit{request push notification} ke layanan APNs.
	\item Implementasi pemantauan aplikasi dapat dilakukan dengan menggunakan fitur \textit{metrics}, \textit{health}, \textit{environment}, dan \textit{logfile} pada pustaka Actuator.
\end{enumerate}

\section{Saran}
\begin{enumerate}
    \item Memperbarui data perangkat yang sudah tidak aktif berdasarkan \textit{response} dari layanan APNs dan FCM ketika mengirim notifikasi.
    \item Menambahkan indeks untuk mempercepat pencarian di sistem basis data.
    \item Menambahkan proses untuk mengulangi pengiriman \textit{push notification} yang ditolak oleh layanan APNs dan FCM karena dianggap \textit{spam} atau karena koneksi terputus.
\end{enumerate}
