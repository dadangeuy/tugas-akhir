\chapter{ANALISIS DAN PERANCANGAN SISTEM}
\par Bab ini membahas analisis dan perancangan sistem untuk mencapai tujuan dari tugas akhir, meliputi perancangan data, proses, dan analisa implementasi secara umum.

\section{Analisis Sistem}
\par Subbab ini membahas hasil analisa perangkat lunak push notification terpusat.
Analisis yang dilakukan meliputi analisis permasalahan, deskripsi umum sistem, dan spesifikasi kebutuhan perangkat lunak.

\subsection{Analisis Permasalahan}
\par Subbab ini membahas permasalahan yang dihadapi oleh aplikasi push notification terpusat saat ini serta penyebab dan dampak dari masalah tersebut.

\subsubsection{Keandalan Antrian Pesan}
\par Arsitektur antrian pesan pada Push Notification Terpusat dapat dilihat pada Gambar \ref{arsitektur_message_queue_lama}. Terdapat beberapa \textit{thread} yang menambahkan dan mengambil pesan dari antrian secara bersamaan. Berdasarkan analisis yang penulis lakukan, struktur data \textit{linked list} yang digunakan oleh antrian tidak mendukung perubahan data secara bersamaan \cite{linkedlist-online}. Konsekuensinya, pesan yang berada di antrian bisa saja hilang tanpa ada \textit{error} yang terdeteksi oleh sistem. Potongan kode implementasi antrian pesan dapat dilihat pada \nameref{lampiran:queue_service}.
\begin{figure}[H]
	\caption{Arsitektur Lama untuk Antrian Pesan} \label{arsitektur_message_queue_lama}
	\includegraphics[width=1\textwidth]{bab3/figures/arsitektur_message_queue_lama.jpg}
\end{figure}

\subsubsection{Durabilitas Penyimpanan Antrian}
\par Antrian \textit{packet} pada Push Notification Terpusat disimpan dalam sebuah struktur data yang ada di sistem. Konsekuensinya, jika sistem \textit{down}, seluruh \textit{packet} yang berada diantrian akan hilang. Selain itu, karena semua pesan disimpan di sistem, sistem akan kehabisan memori dan \textit{crash} jika jumlah pesan yang diantrikan terlalu banyak.

\subsection{Deskripsi Umum Sistem}
\par Push Notification Terpusat merupakan layanan yang digunakan untuk mengirim \textit{push notification} ke perangkat pengguna secara terjadwal dan \textit{asynchronous}. Pada tugas akhir ini, aplikasi akan dibagi menjadi 3 modul, yaitu Scheduler, Sender APN, dan Sender FCM.
\par Secara umum, Scheduler bertanggung jawab untuk menjadwalkan dan mengantrikan \textit{packet}, Sender APN bertanggung jawab untuk mengirimkan \textit{packet} yang diantrikan ke perangkat iOS lewat layanan APNs, dan Sender FCM bertanggung jawab untuk mengirimkan \textit{packet} yang diantrikan ke perangkat Android dan Web lewat layanan FCM.

\subsection{Spesifikasi Kebutuhan Perangkat Lunak}
\par Subbab ini membahas spesifikasi kebutuhan perangkat lunak dari hasil analisis yang telah dilakukan. Subbab ini berisi kebutuhan perangkat lunak yang direpresentasikan dalam bentuk kebutuhan fungsional, kebutuhan non fungsional, dan diagram kasus penggunaan.

\subsubsection{Aktor}
\par Aktor adalah pihak-pihak, baik manusia maupun sistem atau perangkat lain yang terlibat dan berinteraksi secara langsung dengan sistem. Rincian aktor-aktor yang terdapat pada aplikasi push notification terpusat dijelaskan pada Tabel \ref{t:aktor}.
\begin{longtable}{|p{2cm}|p{7cm}|}
    \caption{Aktor pada Sistem} \label{t:aktor} \\ \hline
    \rowcolor{lightgray} Aktor & Tugas \\ \hline
    Scheduler & Membuat dan menjadwalkan pengiriman \textit{packet}. \\ \hline
    Sender APN & Mengirimkan \textit{packet} untuk perangkat iOS. \\ \hline
    Sender FCM & Mengirimkan \textit{packet} notifikasi untuk perangkat Android dan Web. \\ \hline
\end{longtable}

\subsubsection{Kebutuhan Fungsional}
\par Kebutuhan fungsional mendefinisikan hal-hal yang harus dilakukan oleh sistem berdasarkan masukan yang diterima, proses yang dilakukan, dan hasil yang dicapai. Kebutuhan fungsional aplikasi push notification terpusat dijelaskan pada Tabel \ref{t:fungsional}.
\begin{longtable}{|p{1cm}|p{3cm}|p{5cm}|}
    \caption{Kebutuhan Fungsional Sistem} \label{t:fungsional} \\ \hline
    \rowcolor{lightgray} Kode & {Kebutuhan} & Deskripsi \\ \hline
    F-01 & Pembuatan \textit{Packet} & Jika terdapat \textit{batch} yang belum dan boleh dibuatkan \textit{packet}, sistem dapat membuatkan \textit{packet} dari \textit{batch} tersebut. \\ \hline
    F-02 & Menambahkan \textit{Packet} ke Antrian & Jika terdapat \textit{packet} yang belum dan sudah waktunya dikirim, sistem dapat menambahkan \textit{packet} tersebut ke antrian. \\ \hline
    F-03 & Pengiriman \textit{Packet} ke APNs & Jika terdapat \textit{packet} untuk perangkat iOS di antrian, sistem dapat mengambil \textit{packet} dari antrian dan mengirim \textit{packet} ke layanan APNs. \\ \hline
    F-04 & Pengiriman \textit{Packet} ke FCM & Jika terdapat \textit{packet} untuk perangkat Android dan Web di antrian, sistem dapat mengambil \textit{packet} dari antrian dan mengirim \textit{packet} ke layanan FCM. \\ \hline
    F05 & Menampilkan Penggunaan Sumber Daya & Jika mendapat \textit{request} HTTP tertentu, sistem dapat menampilkan penggunaan sumber daya (CPU dan Memori) yang digunakan dalam bentuk \textit{response} HTTP. \\ \hline
    F-06 & Menampilkan Status Kesehatan & Jika mendapat \textit{request} HTTP tertentu, sistem dapat menampilkan kondisi layanan yang berhubungan dengan sistem (sistem basis data dan antrian pesan) dalam bentuk \textit{response} HTTP. \\ \hline
    F-07 & Menampilkan Konfigurasi & Jika mendapat \textit{request} HTTP tertentu, sistem dapat menampilkan konfigurasi sistem dalam bentuk \textit{response} HTTP. \\ \hline
    F-08 & Menampilkan \textit{Log} & Jika mendapat \textit{request} HTTP tertentu, sistem dapat menampilkan isi \textit{log} sistem dalam bentuk \textit{response} HTTP. \\ \hline
\end{longtable}

\subsubsection{Kebutuhan Non Fungsional}
\par Kebutuhan non fungsional mendefinisikan bagaimana sistem harus bekerja dalam kondisi tertentu. Kebutuhan non fungsional aplikasi push notification terpusat dijelaskan pada Tabel \ref{t:non_fungsional}.
\begin{longtable}{|p{1.2cm}|p{2cm}|p{5.5cm}|}
	\caption{Kebutuhan Non Fungsional Sistem} \label{t:non_fungsional} \\ \hline
    \rowcolor{lightgray} Kode & Aspek & Deskripsi \\ \hline
    NF-01 & Performa & Aplikasi dapat mengirimkan 100 ribu \textit{push notification} dalam waktu kurang dari 6 jam. \\ \hline
    NF-02 & Keandalan & Aplikasi dapat mengirimkan 100 ribu \textit{push notification} ke layanan APNs dan FCM tanpa gagal. \\ \hline
    NF-03 & Ketersediaan & Aplikasi dapat menangani pengiriman 100 ribu \textit{push notification} tanpa ada layanan yang \textit{down}. \\ \hline
    NF-04 & Durabilitas & Aplikasi dapat mengirimkan \textit{push notification} saat salah satu layanan dimatikan sementara. \\ \hline
\end{longtable}

\subsubsection{Kasus Penggunaan}
\par Subbab ini membahas kasus penggunaan perangkat lunak secara rinci. Kasus penggunaan dibuat berdasarkan hasil analisis yang pada subbab sebelumnya. Diagram kasus penggunaan untuk setiap modul (Scheduler, Sender APN, dan Sender FCM) aplikasi \textit{push notification} terpusat dapat dilihat pada Gambar \ref{diagram_kasus_penggunaan}.
\begin{figure}[H]
	\caption{Diagram Kasus Penggunaan} \label{diagram_kasus_penggunaan}
    \includegraphics[width=1\textwidth]{bab3/figures/diagram_kasus_penggunaan.jpg}
\end{figure}

% template tabel deskripsi kasus penggunaan
\newcommand\tableUcDesc[8] {
\begin{longtable}{|p{2.5cm}|p{6.5cm}|}
	\caption{Kasus Penggunaan #3} \label{#1} \\ \hline
    \rowcolor{lightgray} Komponen & Deskripsi \\ \hline
    Kode & #2 \\ \hline
    Nama & #3 \\ \hline
    Aktor & #4 \\ \hline
    Kondisi Awal & #5 \\ \hline
    Kondisi Akhir & #6 \\ \hline
    Alur Normal & #7 \\ \hline
    Alur Alternatif & #8 \\ \hline
\end{longtable}
}

\paragraph{Pembuatan Packet}
\par Pada kasus ini, Scheduler akan mencari batch yang belum dan boleh dibuatkan \textit{packet} secara berkala. Jika ada, Scheduler akan membuatkan \textit{packet} untuk \textit{batch} tersebut. Rincian skenario dapat dilihat pada Tabel \ref{t:skenario_pembuatan_packet}.
\tableUcDesc
{t:skenario_pembuatan_packet}
{UC-01}
{Pembuatan \textit{Packet}}
{Scheduler}
{Terdapat \textit{batch} yang belum dan boleh dibuatkan \textit{packet}}
{\textit{Packet} dibuat untuk \textit{batch} tersebut}
{
\begin{enumerate}
    \item Aktor memeriksa apakah terdapat \textit{batch} yang belum dan boleh dibuatkan \textit{packet}.
    \item Aktor membuatkan data \textit{packet} dari batch tersebut.
    \item Aktor menyimpan data \textit{packet} dan memperbarui data \textit{batch}.
\end{enumerate}
}
{-}

\paragraph{Menambahkan Packet ke Antrian}
\par Pada kasus ini, Scheduler akan mencari \textit{packet} yang belum dan sudah waktunya dikirim secara berkala. Jika ada, Scheduler akan menambahkan \textit{packet} tersebut ke antrian pesan dengan topik yang dibagi berdasarkan jenis perangkat penerima. Rincian skenario dapat dilihat pada Tabel \ref{t:skenario_menambahkan_packet_ke_antrian}.
\tableUcDesc
{t:skenario_menambahkan_packet_ke_antrian}
{UC-02}
{Menambahkan \textit{Packet} ke Antrian}
{Scheduler}
{Terdapat \textit{packet} yang belum dan sudah waktunya dikirim}
{\textit{Packet} ditambahkan ke antrian}
{
\begin{enumerate}
    \item Aktor memeriksa apakah terdapat \textit{packet} yang belum dan sudah waktunya dikirim.
    \item Aktor menambahkan \textit{packet} ke dalam antrian dengan pembagian topik berdasarkan tipe target perangkat.
    \item Aktor memperbarui data \textit{packet} (status sedang menunggu).
\end{enumerate}
}
{-}

\paragraph{Pengiriman Packet ke APNs}
\par Pada kasus ini, Sender APN akan menunggu sistem antrian pesan untuk mengirimkan \textit{packet} yang ada di antrian topik iOS. Setelah packet diterima, packet akan dikirimkan ke layanan APNs. Rincian skenario dapat dilihat pada Tabel \ref{t:skenario_pengiriman_packet_ke_apns}.
\tableUcDesc
{t:skenario_pengiriman_packet_ke_apns}
{UC-03}
{Pengiriman \textit{Packet} ke APNs}
{Sender APN}
{Terdapat \textit{packet} di antrian topik iOS}
{\textit{Packet} diterima oleh layanan APNs}
{
\begin{enumerate}
    \item Aktor menerima \textit{packet} dari antrian topik iOS.
    \item Aktor mengirimkan \textit{request push notification} ke layanan APNs berdasarkan data \textit{packet}.
    \item Aktor memperbarui data \textit{packet} berdasarkan \textit{response} dari APNs (berhasil atau gagal dan alasan gagal).
\end{enumerate}
}
{-}

\paragraph{Pengiriman Packet ke FCM}
\par Pada kasus ini, Sender FCM akan menunggu sistem antrian pesan untuk mengirimkan \textit{packet} yang ada di antrian topik Android dan Web. Setelah \textit{packet} diterima, \textit{packet} akan dikirimkan ke layanan FCM. Rincian skenario dapat dilihat pada Tabel \ref{t:skenario_pengiriman_packet_ke_fcm}.
\tableUcDesc
{t:skenario_pengiriman_packet_ke_fcm}
{UC-04}
{Pengiriman \textit{Packet} ke FCM}
{Sender FCM}
{Terdapat \textit{packet} di antrian topik Android atau Web}
{Packet diterima oleh layanan FCM}
{
\begin{enumerate}
    \item Aktor menerima packet dari antrian topik Android atau Web.
    \item Aktor mengirimkan \textit{request push notification} ke layanan FCM berdasarkan data \textit{packet}.
    \item Aktor memperbarui data \textit{packet} berdasarkan \textit{response} dari FCM (berhasil atau gagal dan alasan gagal).
\end{enumerate}
}
{-}

\paragraph{Menampilkan Penggunaan Sumber Daya}
\par Pada kasus ini, Scheduler, Sender APN, dan Sender FCM dapat menampilkan penggunaan sumber daya (CPU dan Memori) yang digunakan aplikasi dengan menggunakan REST API. Rincian skenario dapat dilihat pada Tabel \ref{t:skenario_menampilkan_penggunaan_sumber_daya}.
\tableUcDesc
{t:skenario_menampilkan_penggunaan_sumber_daya}
{UC-05}
{Menampilkan Penggunaan Sumber Daya}
{Scheduler, Sender APN, Sender FCM}
{Aktor menerima \textit{request} HTTP}
{Aktor mengembalikan \textit{response} HTTP}
{
	\begin{enumerate}
		\item Aktor menerima \textit{request} HTTP untuk melihat penggunaan sumber daya.
		\item Aktor memeriksa metrik penggunaan sumber daya dari sistem operasi.
		\item Aktor mengembalikan hasil pemeriksaan penggunaan sumber daya dalam bentuk \textit{response} HTTP.
	\end{enumerate}
}
{-}

\paragraph{Menampilkan Status Kesehatan}
\par Pada kasus ini, Scheduler, Sender APN, dan Sender FCM dapat menampilkan status kesehatan aplikasi dan layanan yang terhubung dengan aplikasi dengan menggunakan REST API. Rincian skenario dapat dilihat pada Tabel \ref{t:skenario_menampilkan_status_kesehatan}.
\tableUcDesc
{t:skenario_menampilkan_status_kesehatan}
{UC-06}
{Menampilkan Status Kesehatan}
{Scheduler, Sender APN, Sender FCM}
{Aktor menerima request HTTP}
{Aktor mengembalikan response HTTP}
{
	\begin{enumerate}
		\item Aktor menerima \textit{request} HTTP untuk melihat status kesehatan.
		\item Aktor memeriksa apakah aktor dapat terhubung ke layanan-layanan yang dibutuhkan oleh aktor untuk beroperasi.
		\item Aktor mengembalikan status (bisa diakses atau tidak) layanan-layanan yang dibutuhkan dalam bentuk \textit{response} HTTP.
	\end{enumerate}
}
{-}

\paragraph{Menampilkan Konfigurasi}
\par Pada kasus ini, Scheduler, Sender APN, dan Sender FCM dapat menampilkan konfigurasi aplikasi dengan menggunakan REST API. Rincian skenario dapat dilihat pada Tabel \ref{t:skenario_menampilkan_konfigurasi}.
\tableUcDesc
{t:skenario_menampilkan_konfigurasi}
{UC-07}
{Menampilkan Konfigurasi}
{Scheduler, Sender APN, Sender FCM}
{Aktor menerima \textit{request} HTTP}
{Aktor mengembalikan \textit{response} HTTP}
{
	\begin{enumerate}
		\item Aktor menerima \textit{request} HTTP untuk melihat konfigurasi aplikasi.
		\item Aktor memeriksa konfigurasi \textit{Spring} yang ada di aplikasi.
		\item Aktor memeriksa \textit{environment variable} yang ada di sistem operasi.
		\item Aktor mengembalikan konfigurasi \textit{Spring} dan \textit{environment variable} yang ada dalam bentuk \textit{response} HTTP.
	\end{enumerate}
}
{-}

\paragraph{Menampilkan Log}
\par Pada kasus ini, Scheduler, Sender APN, dan Sender FCM dapat menampilkan \textit{log} yang dikeluarkan oleh aplikasi dengan menggunakan REST API. Rincian skenario dapat dilihat pada Tabel \ref{t:skenario_menampilkan_log}.
\tableUcDesc
{t:skenario_menampilkan_log}
{UC-08}
{Menampilkan \textit{Log}}
{Scheduler, Sender APN, Sender FCM}
{Aktor menerima \textit{request} HTTP}
{Aktor mengembalikan \textit{response} HTTP}
{
\begin{enumerate}
	\item Aktor menerima \textit{request} HTTP untuk melihat log aplikasi.
	\item Aktor mencari lokasi file \textit{log} yang digunakan aplikasi untuk menyimpan \textit{log}.
	\item Aktor mengembalikan isi file \textit{log} dalam bentuk \textit{response} HTTP.
\end{enumerate}
}
{-}

\section{Perancangan Sistem}
\par Subbab ini akan membahas tahapan perancangan sistem yang dibagi menjadi beberapa bagian, yaitu perancangan arsitektur, basis data, antrian pesan, dan proses.

\subsection{Perancangan Arsitektur} \label{s:perancangan_arsitektur}
\par Push Notification Terpusat akan dibagi menjadi 3 modul, yaitu Scheduler, Sender APN, dan Sender FCM. Aplikasi dibangun dengan bahasa pemrograman Java, dengan kerangka kerja Spring. Aplikasi saling terhubung lewat sistem basis data (SQL Server) dan sistem antrian pesan (Kafka). Secara garis besar, aplikasi ini memiliki rancangan arsitektur yang dapat dilihat pada Gambar \ref{f:arsitektur_aplikasi}.
\begin{figure}[H]
	\caption{Arsitektur Pengiriman Notifikasi} \label{f:arsitektur_aplikasi}
    \includegraphics[width=1\textwidth]{bab3/figures/arsitektur_pengiriman_notifikasi.jpg}
\end{figure}

\subsection{Perancangan Basis Data}
\par Subbab ini membahas bagaimana rancangan basis data yang digunakan pada aplikasi \textit{push notification} terpusat. Sistem basis data yang digunakan adalah SQL Server. Conceptual Data Model dapat dilihat pada \nameref{lampiran:cdm}.

\subsubsection{Tabel User}
\par Tabel user digunakan untuk menyimpan data pengguna aplikasi yang ada di Institut Teknologi Sepuluh Nopember. Rincian atribut dapat dilihat pada Tabel \ref{tabel_user}.
\begin{longtable}{|p{2cm}|p{2.5cm}|p{4.5cm}|}
 	\caption{Tabel User} \label{tabel_user} \\ \hline
    {Atribut} & {Tipe Data} & {Deskripsi} \\ \hline
    User ID & uuid & Primary key tabel \\ \hline
    Name & varchar(150) & - \\ \hline
    Nickname & varchar(20) & - \\ \hline
    Username & varchar(255) & - \\ \hline
    Password & varchar(256) & - \\ \hline
    Email & varchar(255) & - \\ \hline
    Email Verified & numeric(1) & - \\ \hline
    Scope & varchar(4000) & - \\ \hline
    Alternate Email & varchar(255) & - \\ \hline
    Alternate Email Verified & numeric(1) & - \\ \hline
    Phone & varchar(18) & - \\ \hline
    Phone Verified & numeric(1) & - \\ \hline
    Enabled & numeric(1) & - \\ \hline
    Picture & varbinary(max) & - \\ \hline
    Gender & char(1) & - \\ \hline
    Birth Date & date & - \\ \hline
    Zone Info & varchar(40) & - \\ \hline
    Locale & varchar(10) & - \\ \hline
    Integra ID & numeric(12) & - \\ \hline
    Regional ID & varchar(25) & - \\ \hline
    Must Change Password & numeric(1) & - \\ \hline
    Sandbox & numeric(1) & - \\ \hline
    Locked & datetime & - \\ \hline
    Suspended & datetime & - \\ \hline
    Has Suspended & numeric(1) & - \\ \hline
    Group ID & int & - \\ \hline
    Auth Method ID & numeric(2) & - \\ \hline
    Created At & datetime & Tanggal dan waktu dibuat \\ \hline
    Update At & datetime & Tanggal dan waktu diperbarui \\ \hline
\end{longtable}

\subsubsection{Tabel Group}
\par Tabel group digunakan untuk menyimpan data kelompok pengguna. Rincian atribut dapat dilihat pada Tabel \ref{tabel_group}.
\begin{longtable}{|p{2cm}|p{2.5cm}|p{4.5cm}|}
	\caption{Tabel Group} \label{tabel_group} \\ \hline
    {Atribut} & {Tipe Data} & {Deskripsi} \\ \hline
    Group ID & uuid & Primary key tabel \\ \hline
    Name & varchar(100) & - \\ \hline
    Created At & datetime & Tanggal dan waktu dibuat \\ \hline
    Update At & datetime & Tanggal dan waktu diperbarui \\ \hline
\end{longtable}

\subsubsection{Tabel Group Member}
\par Tabel group member digunakan untuk menyimpan data pengguna yang menjadi anggota dari suatu kelompok. Rincian atribut dapat dilihat pada Tabel \ref{tabel_group_member}.
\begin{longtable}{|p{2cm}|p{2.5cm}|p{4.5cm}|}
	\caption{Tabel Group Member} \label{tabel_group_member} \\ \hline
    {Atribut} & {Tipe Data} & {Deskripsi} \\ \hline
    Group ID & uuid & Grup tempat pengguna terdaftar \\ \hline
    User ID & uuid & Pengguna yang terdaftar di grup \\ \hline
    Created At & datetime & Tanggal dan waktu dibuat \\ \hline
    Update At & datetime & Tanggal dan waktu diperbarui \\ \hline
\end{longtable}

\subsubsection{Tabel Client}
\par Tabel client digunakan untuk menyimpan data aplikasi yang ada di Institut Teknologi Sepuluh Nopember. Rincian atribut dapat dilihat pada Tabel \ref{tabel_client}.
\begin{longtable}{|p{2cm}|p{2.5cm}|p{4.5cm}|}
	\caption{Tabel Client} \label{tabel_client} \\ \hline
    {Atribut} & {Tipe Data} & {Deskripsi} \\ \hline
    Client ID & uuid & Primary key tabel \\ \hline
    Client Name & varchar(100) & - \\ \hline
    Client Description & varchar(250) & - \\ \hline
    Client Secret & varchar(255) & - \\ \hline
    Expires At & datetime & - \\ \hline
    Logo & varchar(100) & - \\ \hline
    Redirect URI & varchar(2000) & - \\ \hline
    Post Logout Redirect URIS & varchar(2048) & - \\ \hline
    Front Channel Logout URI & varchar(255) & - \\ \hline
    Front Channel Logout Session Required & numeric(1) & - \\ \hline
    Back Channel Logout URI & varchar(255) & - \\ \hline
    Back Channel Logout Session Required & numeric(1) & - \\ \hline
    Base URI & varchar(255) & - \\ \hline
    API Base URI & varchar(255) & - \\ \hline
    Application Type & char(1) & - \\ \hline
    Contact Name & varchar(255) & - \\ \hline
    Contact Email & varchar(255) & - \\ \hline
    Preauthorized & numeric(1) & - \\ \hline
    Grant Types & varchar(80) & - \\ \hline
    Scope & varchar(4000) & - \\ \hline
    Sandbox & numeric(1) & - \\ \hline
    Is Moderated & numeric(1) & - \\ \hline
    Visible & numeric(1) & - \\ \hline
    Category ID & int & - \\ \hline
    Auth Type ID & char(1) & - \\ \hline
    User ID & uuid & - \\ \hline
    Provider ID & uuid & - \\ \hline
    Created At & datetime & Tanggal dan waktu dibuat \\ \hline
    Update At & datetime & Tanggal dan waktu diperbarui \\ \hline
\end{longtable}

\subsubsection{Tabel Certificate}
\par Tabel certificate digunakan untuk menyimpan data sertifikat yang digunakan untuk autentikasi ke layanan APNs dan FCM. Rincian atribut dapat dilihat pada Tabel \ref{tabel_certificate}.
\begin{longtable}{|p{2cm}|p{2.5cm}|p{4.5cm}|}
	\caption{Tabel Certificate} \label{tabel_certificate} \\ \hline
    {Atribut} & {Tipe Data} & {Deskripsi} \\ \hline
    Certificate ID & uuid & Primary key tabel \\ \hline
    Client ID & uuid & Foreign key tabel client \\ \hline
    Bundle ID & varchar(255) & Bundle ID untuk aplikasi iOS \\ \hline
    Certificate Key & text & File sertifikat yang sudah diencode dengan base64 \\ \hline
    Type & char(1) & Tipe sertifikat (FCM atau APNs) \\ \hline
    Password & varchar(255) & Kata sandi untuk sertifikat iOS \\ \hline
\end{longtable}

\subsubsection{Tabel Device}
\par Tabel device digunakan untuk menyimpan data perangkat pengguna aplikasi yang terdaftar di layanan APNs dan FCM. Rincian atribut dapat dilihat pada Tabel \ref{tabel_device}.
\begin{longtable}{|p{2cm}|p{2.5cm}|p{4.5cm}|}
	\caption{Tabel Device} \label{tabel_device} \\ \hline
    {Atribut} & {Tipe Data} & {Deskripsi} \\ \hline
    Device ID & uuid & Primary key tabel \\ \hline
    Client ID & uuid & Client tempat perangkat terdaftar \\ \hline
    User ID & uuid & User pemilik perangkat \\ \hline
    Device Token & varchar(255) & Token yang terdaftar di layanan APNs dan FCM \\ \hline
    Device Type & char(1) & Jenis perangkat (Android, Web, atau iOS) \\ \hline
    Active & numeric(1) & Perangkat aktif atau tidak \\ \hline
    Registration Date & datetime & - \\ \hline
    Invalidate Date & datetime & - \\ \hline
\end{longtable}

\subsubsection{Tabel Batch}
\par Tabel batch digunakan untuk menyimpan data notifikasi yang akan dikirim ke beberapa perangkat. Rincian atribut dapat dilihat pada Tabel \ref{tabel_batch}.
\begin{longtable}{|p{2cm}|p{2.5cm}|p{4.5cm}|}
	\caption{Tabel Batch} \label{tabel_batch} \\ \hline
    {Atribut} & {Tipe Data} & {Deskripsi} \\ \hline
    Batch ID & uuid & Primary key tabel \\ \hline
    Title & varchar(255) & Judul notifikasi \\ \hline
    Body & varchar(255) & Isi pesan notifikasi \\ \hline
    Image & varchar(255) & Nama atau URL Gambar \\ \hline
    Sound & varchar(255) & Nama atau URL Suara \\ \hline
    Action & varchar(255) & Nama aksi yang dijalankan jika notifikasi dibuka \\ \hline
    Additional Data & varchar(255) & Data tambahan dengan format JSON \\ \hline
    Delivery Date & datetime & Waktu notifikasi dikirim \\ \hline
    Started Date & datetime & Waktu batch mulai diproses \\ \hline
    Finished Date & datetime & Waktu batch selesai diproses \\ \hline
    Is Allowed & numeric(1) & Batch boleh diproses atau tidak \\ \hline
    User Sender ID & uuid & User pembuat batch \\ \hline
    Client Sender ID & uuid & Client pembuat batch \\ \hline
    Client Destination ID & uuid & Client tujuan penerima notifikasi \\ \hline
    Created At & datetime & Tanggal dan waktu dibuat \\ \hline
    Update At & datetime & Tanggal dan waktu diperbarui \\ \hline
\end{longtable}

\subsubsection{Tabel User Destination}
\par Tabel user destination digunakan untuk menyimpan data pengguna yang menjadi target penerima notifikasi dalam satu \textit{batch}. Rincian atribut dapat dilihat pada Tabel \ref{tabel_user_destination}.
\begin{longtable}{|p{2cm}|p{2.5cm}|p{4.5cm}|}
	\caption{Tabel User Destination} \label{tabel_user_destination} \\ \hline
    {Atribut} & {Tipe Data} & {Deskripsi} \\ \hline
    Batch ID & uuid & Batch notifikasi \\ \hline
    User ID & uuid & Pengguna penerima notifikasi \\ \hline
    Created At & datetime & Tanggal dan waktu dibuat \\ \hline
    Update At & datetime & Tanggal dan waktu diperbarui \\ \hline
\end{longtable}

\subsubsection{Tabel Group Destination}
\par Tabel group destination digunakan untuk menyimpan data kelompok yang menjadi target penerima notifikasi dalam satu \textit{batch}. Rincian atribut dapat dilihat pada Tabel \ref{tabel_group_destination}.
\begin{longtable}{|p{2cm}|p{2.5cm}|p{4.5cm}|}
	\caption{Tabel Group Destination} \label{tabel_group_destination} \\ \hline
    {Atribut} & {Tipe Data} & {Deskripsi} \\ \hline
    Batch ID & uuid & Batch notifikasi \\ \hline
    Group ID & uuid & Grup penerima notifikasi \\ \hline
    Created At & datetime & Tanggal dan waktu dibuat \\ \hline
    Update At & datetime & Tanggal dan waktu diperbarui \\ \hline
\end{longtable}

\subsubsection{Tabel Packet} \label{s:tabel_packet}
\par Tabel packet digunakan untuk menyimpan data notifikasi yang dikirim ke satu perangkat. Rincian atribut dapat dilihat pada Tabel \ref{tabel_packet}.
\begin{longtable}{|p{2cm}|p{2.5cm}|p{4.5cm}|}
	\caption{Tabel Packet} \label{tabel_packet} \\ \hline
	{Atribut} & {Tipe Data} & {Deskripsi} \\ \hline
	Packet ID & uuid & Primary key tabel \\ \hline
	Batch ID & uuid & Batch notifikasi \\ \hline
	Device Token ID & uuid & Perangkat penerima notifikasi \\ \hline
	Sent At & datetime & Waktu notifikasi diterima oleh APNs atau FCM \\ \hline
	Reason & varchar(255) & Penyebab jika terjadi kegagalan \\ \hline
	Packet Status & numeric(1) & Status pengiriman packet (dibuat, menunggu, berhasil, atau gagal) \\ \hline
	Created At & datetime & Tanggal dan waktu dibuat \\ \hline
	Update At & datetime & Tanggal dan waktu diperbarui \\ \hline
\end{longtable}

\subsection{Perancangan Antrian Pesan}
\par Subbab ini membahas bagaimana rancangan antrian pesan yang akan digunakan. Sistem antrian pesan yang digunakan adalah Kafka.

\subsubsection{Perancangan Skema}
\par Berdasarkan rancangan arsitektur pada Subbab \ref{s:perancangan_arsitektur}, Scheduler akan mengirimkan \textit{packet} ke Sender APN dan Sender FCM lewat antrian pesan. Pertukaran data \textit{packet} di antrian pesan akan menggunakan format JSON. Skema JSON dapat dilihat pada Kode Sumber \ref{json:packet}.
\lstinputlisting[label=json:packet, caption=Skema JSON Packet, language=SQL] {bab3/json/packet.json}

\subsubsection{Perancangan Topik}
\par Berdasarkan rancangan arsitektur pada Subbab \ref{s:perancangan_arsitektur}, antrian \textit{packet} akan dipisah berdasarkan tipe perangkat penerima \textit{packet}. Rincian pembagian topik dapat dilihat pada Tabel \ref{t:pembagian_topik_antrian}.
\begin{longtable}{|p{2cm}|p{2cm}|p{5cm}|}
	\caption{Pembagian Topik Antrian} \label{t:pembagian_topik_antrian} \\ \hline
	\rowcolor{lightgray} Perangkat & Topik & Deskripsi \\ \hline
	Android & android & Berisi \textit{packet} untuk perangkat Android. \\ \hline
	Web & web & Berisi \textit{packet} untuk perangkat Web. \\ \hline
	iOS & ios & Berisi \textit{packet} untuk perangkat iOS. \\ \hline
\end{longtable}

\subsection{Perancangan Proses}
\par Subbab ini menjelaskan tentang rancangan, tujuan, dan diagram alir proses-proses yang ada pada aplikasi \textit{push notification} terpusat.

\subsubsection{Proses Pembuatan Packet}
\par Proses ini bertujuan untuk membuat \textit{packet} dari \textit{batch} yang baru dibuat. Proses pembuatan dapat dilihat di diagram alir pada Gambar \ref{flowchart_pembuatan_packet}.
\begin{figure}[hb]
	\caption{Diagram Alir Proses Pembuatan Packet} \label{flowchart_pembuatan_packet}
    \centering\includegraphics[width=1\textwidth]{bab3/figures/flowchart_pembuatan_packet.jpg}
\end{figure}

\subsubsection{Proses Menambahkan Packet ke Antrian}
\label{3:proses_menambahkan_packet_ke_antrian}
\par Proses ini bertujuan untuk mengantrikan \textit{packet} yang sudah waktunya untuk dikirim.
Antrian \textit{packet} dibagi berdasarkan jenis perangkat penerima notifikasi (Android, Web, atau iOS). Proses pengantrian dapat dilihat di diagram alir pada Gambar \ref{flowchart_menambahkan_packet_ke_antrian}.
\begin{figure}[hb]
	\caption{Diagram Alir Proses Menambahkan Packet ke Antrian} \label{flowchart_menambahkan_packet_ke_antrian}
    \centering\includegraphics[width=1\textwidth]{bab3/figures/flowchart_menambahkan_packet_ke_antrian.jpg}
\end{figure}

\subsubsection{Proses Pengiriman Packet ke APNs}
\par Proses ini bertujuan untuk mengirimkan \textit{packet} yang ada di antrian topik "ios" ke perangkat pengguna yang berbasis iOS lewat layanan APNs. Proses pengiriman \textit{packet} dapat dilihat di diagram alir pada
Gambar \ref{flowchart_pengiriman_packet_ke_apns}.
\begin{figure}[hb]
	\caption{Diagram Alir Proses Pengiriman Packet ke APNs} \label{flowchart_pengiriman_packet_ke_apns}
    \centering\includegraphics[width=0.6\textwidth]{bab3/figures/flowchart_pengiriman_packet_ke_apns.jpg}
\end{figure}

\subsubsection{Proses Pengiriman Packet ke FCM}
\par Proses ini bertujuan untuk mengirimkan \textit{packet} yang ada di antrian topik "android" dan "web" ke perangkat pengguna yang berbasis Android atau Web lewat layanan FCM. Proses pengiriman \textit{packet} dapat dilihat di diagram alir pada Gambar \ref{flowchart_pengiriman_packet_ke_fcm}.
\begin{figure}[hb]
	\caption{Diagram Alir Proses Pengiriman Packet ke FCM} \label{flowchart_pengiriman_packet_ke_fcm}
    \centering\includegraphics[width=0.6\textwidth]{bab3/figures/flowchart_pengiriman_packet_ke_fcm.jpg}
\end{figure}
